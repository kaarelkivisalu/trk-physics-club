\documentclass[a4paper,11pt,twocolumn]{article}
\usepackage{polyglossia} % Eesti keele tugi
\setdefaultlanguage{estonian}
\usepackage{geometry}% Paigutus
\usepackage{graphicx}% Joonised
\graphicspath{ {images/} }
\usepackage{csquotes}% Eesti jutumärgid \enquote{}
\usepackage{enumitem}% Listid
\usepackage[compact]{titlesec}% Kompaktsed pealkirjad
\usepackage{siunitx}% SI ühikud
\usepackage{tikz}
\usepackage[siunitx]{circuitikz}
\usepackage[final]{microtype}
\usepackage{amsmath}
\usepackage{lmodern}
\geometry{
    paper=a4paper, % Paper size, change to letterpaper for US letter size
    top=0.5cm, % Top margin
    bottom=1cm, % Bottom margin
    left=0.5cm, % Left margin
    right=0.5cm, % Right margin
    foot=0.5cm, % Footer-margin distance
    %showframe, % Uncomment to show how the type block is set on the page
}


%\setlength{\intextsep}{0pt}

\setlength{\parindent}{0cm}% Taandridu pole
\setlength{\parskip}{1em}% Paragraafide vahed
\setlist[itemize]{topsep=0em, partopsep=0em, parsep=0em, itemsep=0.5em}% Itemize spacing

\usepackage{xparse}

% Ülesanded \begin{question}[viide][joonis][joonise suurus] (võib olla ka ainult [viide] või [joonis][joonise suurus])
\newcounter{myproblems}
\NewDocumentEnvironment{question}{o o o}
{\par \refstepcounter{myproblems} \textbf{Ülesanne \themyproblems .} \ignorespaces\IfValueT{#1}{\IfValueTF{#3}{\textbf{(#1)} \ignorespaces}{\IfNoValueT{#2}{\textbf{(#1)} \ignorespaces}}}}
{\IfValueT{#2}{\IfValueTF{#3}{\begin{figure}[h!]\includegraphics[width=#3]{#2.pdf}\centering\vspace{-1em}\end{figure}}{\begin{figure}[h!]\includegraphics[width=#2]{#1.pdf}\centering\vspace{-1em}\end{figure}}}\ignorespacesafterend}

% Alaülesanded
\newenvironment{subquestion}
{\setlength{\parskip}{0pt}\begin{enumerate}[label=\alph*), nolistsep]}
{\end{enumerate}\setlength{\parskip}{1em}\ignorespacesafterend}

% Vihjete jaoks
\newenvironment{hint}[1][Vihje]
{\setlength{\parskip}{0em} \textit{#1}: \ignorespaces}
{\setlength{\parskip}{1em}\ignorespacesafterend}

\newcommand{\pvec}[1]{\vec{#1}\mkern2mu\vphantom{#1}}% Primed vector

% Lahendused jaoks
\usepackage{hyperref}
\newenvironment{solutions}
{\begin{enumerate}[label=\textbf{\arabic*.}, wide]}
{\end{enumerate}}

% Displaystyle valemite paigutus
\makeatletter
\g@addto@macro{\normalsize}{%
    \setlength{\abovedisplayskip}{4pt}
    \setlength{\abovedisplayshortskip}{4pt}
    \setlength{\belowdisplayskip}{4pt}
    \setlength{\belowdisplayshortskip}{4pt}
    }
\makeatother

% \directlua{dofile("DetectUnderfull.lua")}
\tikzset{
    odot/.style={
        circle,
        inner sep=0pt,
        node contents={$\odot$},
        scale=1
    },
    otimes/.style={
        circle,
        inner sep=0pt,
        node contents={$\otimes$},
        scale=1
    }}


\begin{document}
{\huge \textbf{Staatika}\hfill \normalsize {nr 1.0.1}} \\
{Kaarel Kivisalu \hfill 12. september 2018}

\section{Hõõrdumine}
Tüüpiline viga staatikaülesannete lahendamisel on öelda, et hõõrdejõud on $\mu N$. See on vale! Avaldis $\mu N$ annab küll maksimaalse hõõrdejõu väärtuse, kuid üldjuhul on hõõrdejõud selline, et süsteem oleks staatilises tasakaalus.
\begin{question}
Klots hõõrdeteguriga $\mu$ asub kaldpinnal. Millise kaldenurga $\alpha$ korral hakkab klots libisema?
\end{question}
\begin{question}[Lahtine 2006, V7][sta1][5cm]
	Kasti tasasel põhjal asub kuul. Kasti põhi asub nurga all horisontaalsuuna suhtes. Kuuli hoiab tasakaalus kasti seina külge kinnitatud niit, mis on paraleelne kasti põhjaga (vt joonis). Kui suure maksimaalse nurga $ \varphi $ võrra saab kasti kallutada, et kuul oleks veel tasakaalus? Hõõrdetegur kuuli ja kasti vahel on $ \mu $.
\end{question}	

\section{Kolm võrrandit}
Staatika ülesannete puhul on lahenduskäik harilikult standardne: tuleb välja kirjutada igale kehale mõjuvate jõudude tasakaalu tingimus $x$-, $y$- ja vajadusel ka $z$-komponendi jaoks; sageli tuleb neile lisada veel jõumomentide tasakaalu tingimus. Nende võrrandite lahendamine  võib olla üldjuhul keeruline ja nende lihtsustamiseks saab kasutada järgnevaid kavalusi:
\begin{itemize}
	\item Teljestik tuleb valida optimaalne, st nii, et võimalikult paljude jõudude projektsioonid läheksid nulliks. Eriti hea on, kui nulliks lähevad nende jõudude projektsioonid, \textit{mida me ei tea ja mis meid tegelikult ei huvita}.
	\item Jõumomentide võrrandi kirjutamisel on tark valida pöörlemistelg nii, et võimalikult paljude jõudude õlad oleksid nullid. Jällegi on eriti tõhus nullida \textit{“ebahuvitavate” jõudude} momente.
\end{itemize}
Kui on tegemist kahemõõtmelise süsteemiga, siis saab iga keha jaoks välja kirjutada kaks võrrandit jõudude jaoks ($ x $- ja $ y $-komponendid) ning ühe võrrandi jõumomentide jaoks.
\begin{question}
	Varras on asetatud toetuma seinale ja põrandale nii, et puutepunktide kaugused nurgast on $h$ vertikaalsihis ja $s$ horisontaalsihis. Sein on libe ($\mu_0=0$). Missugune peab olema hõõrdetegur $\mu$ varda ja põranda vahel, et varras jääks niisugusesse asendisse ilma libisemata? 
\end{question}

\begin{question}[sta2][3.8cm]
	Kolm ühesugust varrast on ühendatud šarniirselt ning kaks otsmist on šarniirselt kinnitatud horisontaalse lae külge punktides $ A $ ja $ B $. Nende punktide vaheline kaugus on kaks korda suurem, kui varraste pikkus. Šarniirse ühenduse $ C $ külge on riputatud koormismassiga $ m $. Millise minimaalse jõuga on vaja hoida šarniiri $ D $, et süsteem püsiks paigal ja varras $ CD $ oleks horisontaalne?
\end{question}


\section{Toereaktsiooni ja hõõrdejõu resultant}
\begin{question}
	Millist minimaalset jõudu on vajalik rakendada selleks, et nihutada paigalt kaldpinnal lebavat klotsi, mille mass on $ m $, kui hõõrdetegur on $ \mu $ ja kaldenurk $ \alpha $? Vaadelda juhtumeid kui
	\begin{subquestion}
		\item $ \alpha=0 $;
		\item $ 0 <\alpha < \arctan{\mu} $.
	\end{subquestion}
\end{question}

\section{Jõudude sihtide kohtumispunkt}
\begin{question}[Piirk 2004, G8]
Sein ja põrand on tehtud samast materjalist. Milline peab olema hõõrdetegur, et $ \alpha =45^\circ $ nurga all vastu seina toetatud ühtlase läbimõõduga ja ühtlase massijaotusega pulk ei hakkaks libisema?
\end{question}
\begin{question}
	Silindrit raadiusega $ R $ tõmmatakse horisontaalse jõuga üle trepiastme kõrgusega h. Milline peab olema hõõrdetegur, et silinder libisema ei hakkaks?
\end{question}

\section{Mitteinertsiaalne taustsüsteem}
Kiirendusega liikuvas süsteemis pole jõudude summa null, järelikult ei kehti ka staatika tingimused. Siin aitab aga üks nipp: tuleb minna kiirendusega liikuvasse (ehk mitteinertsiaalsesse) taustsüsteemi, kus igale kehale hakkab mõjuma inertsiaaljõud $ -m\vec{a} $. Nüüd on keha paigal ja tänu lisatud jõududele kehtivad ka staatika tingimused.
\begin{question}[Lõppv 2014, G7]
	Leidke esirattaveolise sõiduauto maksimaalne kiirendus. Auto mass on $ m $, esi- ja tagarataste telgede vahe $ b $, masskeskme kõrgus $ h $ ning masskeskme horisontaalne kaugus tagateljest $ s $. Hõõrdetegur rataste ja maa vahel on $ \mu $.
\end{question}
\begin{question}[Piirk 2014, G7]
	Horisontaalsel laual asuva klotsi massiga $ m_1 $ peale on asetatud teine klots massiga $ m_2 $. Kahe klotsi vaheline seisuhõõrdetegur on $ \mu_2 $. Alumise klotsi ja laua vaheline liugehõõrdetegur on $ \mu_1 $. Leidke	maksimaalne horisontaalne jõud $ F $, millega võib alumist klotsi tõmmata, ilma et ülemine klots libiseks.
\end{question}

\section{Keerulisemad ülesandeid}	
\begin{question}[sta3][2.2cm]
	Kerge traatvarda üks ots on keeratud rõngaks raadiusega $r$. Varda sirge osa pikkus on $l$ ja teise otsa külge on kinnitatud kuulike massiga $M$. Sel viisil moodustatud pendel on riputatud rõnga abil pöörleva võlli külge. Hõõrdetegur võlli ja rõnga vahel on $\mu$. Leida millise nurga moodustab varras tasakaaluasendis vertikaalsihiga.
\end{question}
\begin{question}[200PPP, P10]
	Victor Hugo novellis \textit{\enquote{Les Misérables}} täheldati, et peategelane Jean Valjean, põgenenud vang, suutis üles ronida nurgast, mille moodustasid kaks ristuvat seina. Leia minimaalne jõud, millega ta pidi seinu lükkama ronimise ajal. Milline on minimaalne seisuhõõrdetegur $ \mu $ sellise vägiteoga hakkama saamiseks?
\end{question}
\begin{question}[Lõppv 2017, G9]
	Kaks jäika traadijuppi pikkusega $L$ on ühendatud otsapidi (nt niidiga seotud) nii, et nende otspunktid on kontaktis ja nende vaheline nurk saab takistuseta muutuda, moodustades V-kujulise figuuri. See traadist moodustis asetatakse horisontaalse libedapinnalise silindri peale nõnda, et tasakaaluasendis moodustub traadist "katus" (tagurpidi "V") tipunurgaga $\alpha$. Massijaotus traadis on ühtlane, hõõre traadi ja silindri vahel puudub.
	\begin{subquestion}
		\item Milline on silindri raadius $R$?
		\item Milline võrratus peab olema rahuldatud, et see asend oleks stabiilne (uurida stabiilsust vaid "katuse" kui terviku pöördumise suhtes eeldades, et traatidevaheline nurk ei muutu)? 
	\end{subquestion}
\end{question}
\begin{question}[E-S 2013, P6]
	Kaldpinnal nurgaga $ \alpha $ lebavad kera ja silinder, kummagi mass on $ m $ ning raadius $ r $. Kehad lastakse lahti võrdselt algkõrguselt $ h $. Kera ja silindri inertsimomendid on vastavalt $ I_{sph}=\frac{2}{5}mr^2 $ ja $ I_{cyl}=\frac{1}{2}mr^{2} $. Kaldpinna ja kehade vaheline hõõrdetegur on $ \mu $.
	\begin{subquestion}
		\item Kumb keha jõuab varem alla? Milline on teise keha suhteline hilinemine $ \gamma = (t_2 − t_1)/t_1 $? Ajad $ t_1 $ ja $ t_2 $ tähistavad vastavalt esimesena ja teisena alla jõudnud kehade liikumise aegu. Eeldage, et veeremine toimub ilma libisemata.
		\item Leidke kaldenurga minimaalne väärtus $ \alpha_0 $, millest alates hakkab silinder lisaks veeremisele ka libisema.
		\item Kui $ \alpha \rightarrow  90^\circ$, siis kaotavad	kehad mõistagi pinnaga kontakti ning jõuavad vabalt langedes võrdse ajaga alla. Milline on aga minimaalne kaldenurk $ \alpha_m $, mille puhul nii kera kui silinder võrdse ajaga alla jõuavad?
	\end{subquestion}
\end{question}
\begin{question}[E-S 2010, P4]
	Massiivset sfäärilist kera massiga $ M = 100 $ kg püütakse veeretada üles mööda vertikaalset seina, rakendades jõudu $ F $ mingisse punkti $ P $ kera pinnal. Hõõrdetegur kera ja seina vahel on $  \mu = 0.7 $.
	\begin{subquestion}
		\item Millise minimaalse jõu $ F_{min} $ abil on võimalik see eesmärk
		saavutada?

		\item Konstrueerige geomeetriliselt seina ja kera külgvaates see punkt $ P $, kuhu niisugune minimaalne jõud peab olema rakendatud, koos rakendatava jõu suunaga.
	\end{subquestion}
\end{question}
\begin{question}[Lõppv 1998, G9][sta4][6cm]
	Õõnes silinder massiga $ m $ ja raadiusega $ R $ seisab horisontaalsel alusel (joonisel on näidatud pealtvaade); tema alumine serv on sile ja kõikjal kontaktis aluspinnaga. Silindrile on mähitud niit, mille vaba otsa tõmmatakse konstantse kiirusega $ v $ piki niidi sihti. Leidke, millise kiirusega liigub silinder. Vaadelda eraldi kahte juhtumit:
	\begin{subquestion}
		\item aluspinna ja silindri vaheline hõõrdetegur on kõikjal null, välja arvatud kitsas sirge vööt (hulga kitsam silindri raadiusest) hõõrdeteguriga $ \mu $, mis on niidiga paralleelne ja mille kaugus niidist $ a < 2R $
		\item aluspinna hõõrdetegur on kõikjal konstantne ning võrdne $ \mu $-ga. 
	\end{subquestion}\setlength{\parskip}{0pt}
	\textit{Vihje: kõva keha suvaline liikumine on vaadeldav pöörlemisena ümber hetkelise pöörlemistelje, s.o. keha iga punkti kiirusvektor on täpselt sama nagu siis, kui
	hetkeline telg oleks tõeliseks pöörlemisteljeks.}
\end{question}
\begin{question}[Lõppv 2015, G9][sta5][6cm]
	Horisontaalpinnal lebab hantel, mis koosneb kaalutust vardast pikkusega $ l = 4a $ ning selle otstele kinnitatud kahest ühesuguse massi ja hõõrdeteguriga väikesest klotsist. Varda külge kaugusele $ a $ ühest klotsist on seotud pikk niit. Algul on niidi suund horisontaalne ja risti vardaga. Niiti aeglaselt tõmmates hakkab hantel pöörduma, sest alguses nihkub vaid üks klots. Milline on nurk $ \alpha $ varda ja niidi vahel siis, kui ka teine klots nihkuma hakkab? 
\end{question}

\end{document}
