\documentclass[a4paper,11pt,twocolumn]{article}
\usepackage{polyglossia} % Eesti keele tugi
\setdefaultlanguage{estonian}
\usepackage{geometry}% Paigutus
\usepackage{graphicx}% Joonised
\graphicspath{ {images/} }
\usepackage{csquotes}% Eesti jutumärgid \enquote{}
\usepackage{enumitem}% Listid
\usepackage[compact]{titlesec}% Kompaktsed pealkirjad
\usepackage{siunitx}% SI ühikud
\usepackage{tikz}
\usepackage[siunitx]{circuitikz}
\usepackage[final]{microtype}
\usepackage{amsmath}
\usepackage{lmodern}
\geometry{
    paper=a4paper, % Paper size, change to letterpaper for US letter size
    top=0.5cm, % Top margin
    bottom=1cm, % Bottom margin
    left=0.5cm, % Left margin
    right=0.5cm, % Right margin
    foot=0.5cm, % Footer-margin distance
    %showframe, % Uncomment to show how the type block is set on the page
}


%\setlength{\intextsep}{0pt}

\setlength{\parindent}{0cm}% Taandridu pole
\setlength{\parskip}{1em}% Paragraafide vahed
\setlist[itemize]{topsep=0em, partopsep=0em, parsep=0em, itemsep=0.5em}% Itemize spacing

\usepackage{xparse}

% Ülesanded \begin{question}[viide][joonis][joonise suurus] (võib olla ka ainult [viide] või [joonis][joonise suurus])
\newcounter{myproblems}
\NewDocumentEnvironment{question}{o o o}
{\par \refstepcounter{myproblems} \textbf{Ülesanne \themyproblems .} \ignorespaces\IfValueT{#1}{\IfValueTF{#3}{\textbf{(#1)} \ignorespaces}{\IfNoValueT{#2}{\textbf{(#1)} \ignorespaces}}}}
{\IfValueT{#2}{\IfValueTF{#3}{\begin{figure}[h!]\includegraphics[width=#3]{#2.pdf}\centering\vspace{-1em}\end{figure}}{\begin{figure}[h!]\includegraphics[width=#2]{#1.pdf}\centering\vspace{-1em}\end{figure}}}\ignorespacesafterend}

% Alaülesanded
\newenvironment{subquestion}
{\setlength{\parskip}{0pt}\begin{enumerate}[label=\alph*), nolistsep]}
{\end{enumerate}\setlength{\parskip}{1em}\ignorespacesafterend}

% Vihjete jaoks
\newenvironment{hint}[1][Vihje]
{\setlength{\parskip}{0em} \textit{#1}: \ignorespaces}
{\setlength{\parskip}{1em}\ignorespacesafterend}

\newcommand{\pvec}[1]{\vec{#1}\mkern2mu\vphantom{#1}}% Primed vector

% Lahendused jaoks
\usepackage{hyperref}
\newenvironment{solutions}
{\begin{enumerate}[label=\textbf{\arabic*.}, wide]}
{\end{enumerate}}

% Displaystyle valemite paigutus
\makeatletter
\g@addto@macro{\normalsize}{%
    \setlength{\abovedisplayskip}{4pt}
    \setlength{\abovedisplayshortskip}{4pt}
    \setlength{\belowdisplayskip}{4pt}
    \setlength{\belowdisplayshortskip}{4pt}
    }
\makeatother

% \directlua{dofile("DetectUnderfull.lua")}
\tikzset{
    odot/.style={
        circle,
        inner sep=0pt,
        node contents={$\odot$},
        scale=1
    },
    otimes/.style={
        circle,
        inner sep=0pt,
        node contents={$\otimes$},
        scale=1
    }}


\begin{document}
{\huge \textbf{Kinemaatika} \hfill \normalsize {nr 3.0.0}} \\
{Kaarel Kivisalu \hfill 3. oktoober 2018}

\section{Kiirused}
Tuleb valida sobiv taustsüsteem. Potentsiaalselt kasulikud taustsüsteemid on, kus
{\setlength{\parskip}{0em}
\begin{itemize}[noitemsep,nolistsep]
	\item mingid kehad on paigal;
	\item mõned kiiruste projektsioonid kaovad;
	\item liikumine on sümmeetriline.
\end{itemize}}
\begin{question}[Piirk 2015, P2]
	Just sel hetkel, kui jõesadamast triivib mööda parv, alustab sadamast pärivoolu liikumist mootorpaat, mis suundub mööda jõge $ s = 21 $ km kaugusel olevasse külasse. Paat jõuab sinna $ t = 45 $ minutiga, pöördub kohe tagasi ja kohtab parve $ l=15 $ km kaugusel külast. Kui suur on voolu kiirus $ v_v $ jões ja paadi kiirus $ v_p $?
\end{question}
\begin{question}[Piirk 2014, P9][kin1][4.4cm]
	Kaks lennukit lendavad samal kõrgusel kiirustega $ v_1 = 800 $ km/h ja $ v_2 = 600 $ km/h. Vaadeldaval hetkel on lennukite liikumise sihid omavahel risti ning kumbki lennuk paikeb sihtide ristumispunktist kaugusel $ a = 20 $ km. Leidke, milline on lennukite vähim vahekaugus järgneva liikumise jooksul, kui eeldada, et kumbki lennuk kurssi ei muuda.
\end{question}
\begin{question}[Kvant F833, NSVL ol. 1983, 8. kl]
	Transportööri lindi peale tõugatakse klots. Lint liigub kiirusega $ v_0= 1 $ m/s; klotsi algkiirus $ u_0= 2 $ m/s on risti lindi kiirusega. Milline on klotsi edaspidise liikumise käigus tema minimaalne kiirus maapinna suhtes? Hõõrdetegur on nii suur, et klots jääb kenasti lindi peale püsima. \textit{Vihje: lühim tee punktist tasandini on risti tasandiga.}
\end{question}
Kui on tegemist \textit{elastse põrkega}, siis on harilikult kasulik vaadelda protsessi \textit{massikeskme süsteemis}.

Kui elastne pall põrkub seinaga, mis liigub kiirusega $ \vec{u} $ pinnanormaali suunas, siis palli kiiruse $ \vec{v} $ normaalkomponent $ \vec{v}_n $ suureneb $ 2\vec{u} $ võrra, s.o $\pvec{v}'_n=-\vec{v}_n+2\vec{u} $.
\begin{question}[NSVL ol. 1983, 9. kl]
	Tennise pall kukub kiirusega $ v $ raske reketi peale ja põrkub sealt elastselt tagasi. Millise kiirusega $ u $ peab liikuma reket selleks, et pall liiguks täisnurga all oma esialgse trajektoori suhtes ning ei hakkaks pöörlema, kui ta ei teinud seda enne põrget? Millise nurga $ \beta $ moodustab seejuures kiirus $ \vec{u} $ reketi tasandi normaalsihi suhtes, kui kiirusvektori $ \vec{v} $ jaoks on see nurk $ \alpha $?
\end{question}

\section{Kiirendused ja nihked}
\begin{question}[kin2][6.5cm]
	Kaks siledat renni asuvad ühes ja samas vertikaaltasandis ning moodustavad horisondiga nurga $ \alpha $ (vt. joon.). Punktidest $ A $ ja $ B $ lastakse ühel ja samal hetkel lahti kaks kuulikest, mis hakkavad renne mööda alla libisema. Esimesel kuulikesel, mis startis punktist $ A $ kulus maapinnale jõudmiseks aega $ t_1 $-jagu; teisel kuulikesel oli laskumise aeg võrdne $ t_2 $-ga. Millisel ajahetkel oli kuulikeste vaheline kaugus kõige väiksem? \textit{Vihje: võib olla kasulik minna mitteinertsiaalsesse taustsüsteemi.}
\end{question}

Kui on tegemist elastsete põrgetega (või põrgetega ilma hõõrdumiseta) tasapinnalt, siis harilikult osutub, et liikumine tasapinnas ja sellega ristuvas sihis on üksteisest sõltumatud. Sellisel juhul on tõhusaks meetodiks nende erisihiliste liikumiste eraldi vaatlemine.
\begin{question}
	Elastne kuulike lastakse lahti kaldpinna (kaldenurk on $ \alpha $) kohal kaugusel $ d $ kaldpinnast. Milline on esimese ja teise põrkepunkti vaheline kaugus. Põrked on hõõrdevabad.
\end{question}

Kui jõud on risti liikumissuunaga (pinna rõhumisjõud libisemisel mööda kõverat pinda, nööri pinge keha liikumisel jäigalt kinnitaud venimatu nööri otsas, magnetväljas laengule mõjuv jõud), siis kiiruse vektor ainult pöördub, tema moodul ei muutu.
\begin{question}[NSVL ol. 1981, 8. kl][kin3][6.8cm]
	Litter libiseb jäise kaldpinna peale, mille kaldenurk on $ \alpha $. Kaldpinna serva ja litri algkiiruse $ v_0=10 $ m/s vaheline nurk on $ \beta = 60^\circ $
	. Litter jättis kaldpinnale sellise jälje, nagu toodud joonisel (seal on kujutatud vaid osa trajektoorist). Leidke nurk $ \alpha $ eeldusel, et hõõrdumisega võib mitte arvestada ja et kaldpinnale
	libisemine oli sujuv.
\end{question}
\begin{question}
	Kolm kilpkonna asuvad alghetkel võrdkülgse kolmnurga tippudes \( 1 \) m kaugusel üksteisest. Nad liiguvad kontsantse kiirusega $ 10 $ cm/s sel moel, et esimene hoiab kogu aeg kurssi teise suunas, teine — kolmanda suunas ja kolmas — esimese suunas. Millise ajavahemiku pärast nad kohtuvad?
\end{question}

\section{Optimaalsed trajektoorid}
Kinemaatikaülesannetes, kus kiirused on antud erinevates keskkondades ja küsitakse kiireimat teed punktist A punkt B, võivad olla abiks \textit{Fermat' printsiip} (formuleeritud geomeetrilise optika jaoks), \textit{Huygensi printsiip} ja \textit{Snelli seadus}.
\begin{question}[kin4][3.8cm]
	Poiss elab lahe $ MOP $ kaldalõigul $ OP $ (vt. joonis) Lahe kaks kallast moodustavad nurga $ \alpha $. Poisi maja asub punktis $ A $, mille kaugus kaldast on $ h $ ja lahesopist $ O $ on $\sqrt{h^2+l^2} $. Poiss püüab kala kaldal $ OM $. Millisel kaugusel $ x $ punktist $ O $ peaks asuma kalapüügikoht, et kodust sinna jõudmiseks kuluks võimalikult vähe aega ja kui pikk on see aeg? Poiss liigub kuival maal kiirusega $ v $ ja paadis kiirusega $ u $.
\end{question} 
\begin{question}[kin5][5.3cm]
	Jões, mille voolukiirus on $ w $ asub punktis $ A $ (kaugus kaldast on $ a $) poiss. Piki kallast jookseb ta kiirusega $ v $ ja ujub kiirusega $ u $; vesi voolab jões kiirusega $ w>u $. Poiss tahab jõuda jõe kaldal ülesvoolu asuvasse punkti $ C $ minimalse ajaga. Millisel kaugusel $ x $ punktiga $ A $ kohakuti asuvast punktist $ B $ peaks veest välja ronima?
\end{question}
\begin{question}
	Milline on see ruumiosa $ \mathcal{R} $, kuhu saab lasta kahuriga, mis asub koordinaatide alguspunktis ja mis annab kuulile algkiiruse $ v_0 $? Laskesuuna võib valida vastavalt vajadusele.
\end{question}
\begin{question}
	Eelmise ülesannete eelduste korral ja teades, et ruumiosa $ \mathcal{R} $ piirjoon on parabool, näita, et kahur asub parabooli fookuses.
\end{question}
\begin{question}
	Millise minimaalse algkiiruse peab andma kivile, et visata üle viilkatuse? Viilkatuse laius on $ b $, ühe otsa kõrgus on $ a $, teise kõrgus on $ c $.
\end{question}

\section{Jäigad kehad, šarniirid ja köied}
\begin{question}[kin6][5cm]
	Jäik kamakas on surutud kahe plaadi vahele, millest üks liigub kiirusega $ v_1 $ ja teine kiirusega $ v_2 $. Antud hetkel on kiirused horisontaalsed ning kamaka ja plaatide puutepunktid kohakuti. Märkige skemaatilisel joonisel kõik need kamaka punktid, mille kiiruse moodul võrdub $ v_1 $- või $ v_2 $-ga.
\end{question}

Kõva keha liikumist on alati võimalik esitada kui pöörlemist ümber hetkelise pöörlemiskeskme.
\begin{question}
	Tsükloid on joon, mida saab defineerida punkti, mis on märgitud veereva veljele, mille raadius on $ R $, trajektoorina. Leia sellise joone kõverusraadius kõrgeimas puntis.
\end{question}
\begin{question}[kin7][5cm]
	Šarniirne konstruktsioon koosneb kahest lülist pikkusega $ 2l $. Tema üks ots on kinnitatud seina külge, teine aga liigub kaugusel $ 3l $ seinast konstantse vertikaalse kiirusega $ v $. Leida šarniirse ühenduspunkti kiirendus sel hetkel, kui 
	\begin{subquestion}
		\item seinapoolne lüli on horisontaalne;
		\item ühenduspunkti kiirus on null.
	\end{subquestion}
\end{question}
\vspace{1em}
\begin{question}[kin8][6cm]
	Silindrile on mähitud niit , mille teine ots on kinnitatud seina külge. Silinder asub horisontaalsel alusel, mida tõmmatakse horisontaalsuunalise kiirusega $ v $ (silindri telje suhtes risti). Leidke silndri telje kiirus sõltuvuses niidi ja vertikaalsihi vahelisest nurgast $ \alpha $. Silinder veereb alusel ilma libisemiseta
\end{question}

\end{document}