\documentclass[a4paper,11pt,twocolumn]{article}
\usepackage{polyglossia} % Eesti keele tugi
\setdefaultlanguage{estonian}
\usepackage{geometry}% Paigutus
\usepackage{graphicx}% Joonised
\graphicspath{ {images/} }
\usepackage{csquotes}% Eesti jutumärgid \enquote{}
\usepackage{enumitem}% Listid
\usepackage[compact]{titlesec}% Kompaktsed pealkirjad
\usepackage{siunitx}% SI ühikud
\usepackage{tikz}
\usepackage[siunitx]{circuitikz}
\usepackage[final]{microtype}
\usepackage{amsmath}
\usepackage{lmodern}
\geometry{
    paper=a4paper, % Paper size, change to letterpaper for US letter size
    top=0.5cm, % Top margin
    bottom=1cm, % Bottom margin
    left=0.5cm, % Left margin
    right=0.5cm, % Right margin
    foot=0.5cm, % Footer-margin distance
    %showframe, % Uncomment to show how the type block is set on the page
}


%\setlength{\intextsep}{0pt}

\setlength{\parindent}{0cm}% Taandridu pole
\setlength{\parskip}{1em}% Paragraafide vahed
\setlist[itemize]{topsep=0em, partopsep=0em, parsep=0em, itemsep=0.5em}% Itemize spacing

\usepackage{xparse}

% Ülesanded \begin{question}[viide][joonis][joonise suurus] (võib olla ka ainult [viide] või [joonis][joonise suurus])
\newcounter{myproblems}
\NewDocumentEnvironment{question}{o o o}
{\par \refstepcounter{myproblems} \textbf{Ülesanne \themyproblems .} \ignorespaces\IfValueT{#1}{\IfValueTF{#3}{\textbf{(#1)} \ignorespaces}{\IfNoValueT{#2}{\textbf{(#1)} \ignorespaces}}}}
{\IfValueT{#2}{\IfValueTF{#3}{\begin{figure}[h!]\includegraphics[width=#3]{#2.pdf}\centering\vspace{-1em}\end{figure}}{\begin{figure}[h!]\includegraphics[width=#2]{#1.pdf}\centering\vspace{-1em}\end{figure}}}\ignorespacesafterend}

% Alaülesanded
\newenvironment{subquestion}
{\setlength{\parskip}{0pt}\begin{enumerate}[label=\alph*), nolistsep]}
{\end{enumerate}\setlength{\parskip}{1em}\ignorespacesafterend}

% Vihjete jaoks
\newenvironment{hint}[1][Vihje]
{\setlength{\parskip}{0em} \textit{#1}: \ignorespaces}
{\setlength{\parskip}{1em}\ignorespacesafterend}

\newcommand{\pvec}[1]{\vec{#1}\mkern2mu\vphantom{#1}}% Primed vector

% Lahendused jaoks
\usepackage{hyperref}
\newenvironment{solutions}
{\begin{enumerate}[label=\textbf{\arabic*.}, wide]}
{\end{enumerate}}

% Displaystyle valemite paigutus
\makeatletter
\g@addto@macro{\normalsize}{%
    \setlength{\abovedisplayskip}{4pt}
    \setlength{\abovedisplayshortskip}{4pt}
    \setlength{\belowdisplayskip}{4pt}
    \setlength{\belowdisplayshortskip}{4pt}
    }
\makeatother

% \directlua{dofile("DetectUnderfull.lua")}
\tikzset{
    odot/.style={
        circle,
        inner sep=0pt,
        node contents={$\odot$},
        scale=1
    },
    otimes/.style={
        circle,
        inner sep=0pt,
        node contents={$\otimes$},
        scale=1
    }}


\begin{document}
{\huge \textbf{Tuletised, diferentsiaalid ja integraalid füüsikas} \hfill \normalsize {nr 4.0.0}} \\
{Kaarel Kivisalu \hfill 31. oktoober 2018}

\section{Tuletis ja diferentsiaal}
Funktsiooni \( f(x) \) diferentsiaaliks nimetame muutu \( df = f(x + dx) − f(x) \). See vastab funktsiooni muutumisele, kui argument muutub \( dx \) võrra. Suhet \( f' = df /dx \) nimetame funktsiooni tuletiseks ning see vastab ka graafiku tõusule \( \tan(\alpha) \) (ekstreemumpunktides on graafiku tõus ja tuletis null). Võib ka vastupidi öelda: diferentsiaali saab avaldada tuletise \( f' \) ja argumendi muudu \( dx \) abil: \( f(x + dx) = f(x) + df \), kus \( df = f'dx \). (Kui valemid tunduvad keerulised, mõtleme graafikute peale).

\begin{table}[h]
	\centering
	\caption{Põhifunktsioonide diferentsiaalid ja tuletised}
	\begin{tabular}{ c | c | c }
		\hline \hline
		Algseos & Diferentsiaal & Tuletis \\ 
		\hline
		\( f(t)=const \) & \( df=0 \) & \( df/dt=0 \) \\  
		\( v=at \) & \( dv=a\ dt \) & \( \dot{v}=a \) \\
		\( x=at^2/2 \) & \( dx=at\ dt \) & \( \dot{x}=at \) \\
		\( f(x)=x^n \) & \( df=nx^{n-1}\ dx \) & \( f'=df/dx=nx^{n-1} \) \\
		\( y=\sin \alpha \) & \( dy=\cos \alpha\ d\alpha \) & \( dy/d\alpha=\cos \alpha \) \\  
		\( y=\cos \alpha \) & \( dy=-\sin \alpha\ d\alpha \) & \( dy/d\alpha=-\sin \alpha  \) \\
		\( y=e^x \) & \( dy=e^x\ dx \) & \( dy/dx=e^x \) \\ 
		\( y=\ln x \) & \( dy=1/x\ dx \) & \( dy/dx=1/x \) \\ 
	\end{tabular}
\end{table}

Liitfunktsiooni tuletise võtmise reegel (ingl k \textit{chain rule}), kus \( f(g) \) ja \( g(x) \):
\[ \frac{df}{dx}=\frac{df}{dg}\frac{dg}{dx} \textrm{.}\]

\begin{table}[h!]
	\centering
	\caption{Liitfunktsioonide diferentsiaalid ja tuletised}
	\begin{tabular}{ c | c | c }
		\hline \hline
		Algseos & Diferentsiaal & Tuletis \\ 
		\hline
		\( f=e^{-\lambda t} \) & \( df=-\lambda e^{-\lambda t}\ dt \) & \( df/dt=-\lambda e^{-\lambda t} \) \\  
		\( x=\cos (\omega t) \) & \( dx=-\omega\sin (\omega t)\ dt\) & \( dx/dt=-\omega\sin (\omega t) \) \\
	\end{tabular}
\end{table}

Mitme muutuja funktsioonide puhul koosneb kogudiferentsiaal osadiferentsiaalide summast.

\begin{table}[h!]
	\centering
	\caption{Mitme muutuja funktsioonide diferentsiaalid}
	\begin{tabular}{ c | c}
		\hline \hline
		Algseos & Diferentsiaal \\ 
		\hline
		\( f=uv \) & \( df=u\ dv+v\ du \) \\  
		\( f=pV^\gamma \) & \( df=\gamma pV^{\gamma-1} dV +V^\gamma dp \) \\
		\( \ln f=\ln p +\gamma \ln V \) & \( d\ln f= dp/p+\gamma dV/V \)
	\end{tabular}
\end{table}

Tavaliselt lõputult kohal 0 diferentseeruva funktsiooni saame kirja panna Maclaurini reana (Taylori rea erijuht): 
\[ f(x)=\sum_{n=0}^{\infty}\frac{f^{(n)} (0)}{n!} x^{n} = f(0) + f'(0) x + \frac{f''(0)}{2} x^2 +...\]

Sellest järeldub, et kui \( x\ll 1 \), siis 
\[ (1+x)^n=1+nx, \qquad \sin x=\tan x=x, \qquad \cos x=1-\frac{x^2}{2} \] 
\[ \ e^x=1+x, \qquad \ln (1+x)=x. \]

\section{Ülesandeid tuletise kohta}
\begin{question}
	Vaatleme harmoonilist võnkumist \( x(t) = A \cos(ωt) \). Avaldage \( v(t) \) ja \( a(t) \). Veenduge, et kogu võnkumise kestel kehtib energia jäävuse seadus. Milline seos kehtib \( a \) ja \( x \) vahel?
\end{question}
\begin{question}
	Vedru jäikusega \( k \) külge on kinnitatud koormis \( m \), avaldage võnkumise sagedus \( f \).
\end{question}
\begin{question}
	Avaldage a) matemaatilise (punktmass) ja b) füüsikalise pendli (meenutage inertsimomenti ja Steineri teoreemi) võnkumise sagedus.
\end{question}
\begin{question}
	Pingeallikale \( \mathcal{E} \) sisetakistusega \( r \) ühendatakse väline tarbija takistusega \( R \). Millise \( R \) korral on tarbijal eralduv võimsus maksimaalne?
\end{question}
\begin{question}
	On teada, et kera ruumala avaldub kujul \( V=\frac{4}{3}\pi r^3 \)
	(miks?). Tuletage sfääri pindala valem.
\end{question}
\begin{question}
	Varras toetub ühe oma otsaga vastu põrandat ning teisega vastu vertikaalset seina. Milline on varda alumise otsa kiirus \( u \) hetkel, mil tema ülemine ots libiseb allapoole kiirusega \( v \) ning nurk põranda ja varda vahel \( \alpha \)?
\end{question}
\begin{question}[Lõppv 2004, G7][mat1][\columnwidth]
	Elektrikannus soojendatakse vett. Teatud hetkel pandi kannu \( T_0 = 0\ ^\circ\)C juures olevat jääd. Joonisel on toodud vee temperatuuri sõltuvus ajast. Kui suur oli jää mass, kui kannu võimsus \( P = 1 \) kW. Jää sulamissoojus on \( L = 335 \) kJ/K. Toatemperatuur \( T_1 = 20\ ^\circ \)C.
\end{question}
\begin{question}
	Koer ajab taga rebast, kes jookseb sirgjoones konstantse kiirusega \( v \). Koera kiiruse moodul on samuti \( v \), kuid vektor \( \vec{v} \) on igal ajahetkel suunatud otse rebase poole. Alghetkel, mil koer rebast märkas ning teda jälitama asus, oli nende vahemaa \( L \) ning koera ja rebase kiirsvektorid omavahel risti. Leidke koera ja rebase minimaalne vahemaa jälitamise ajal.
\end{question}
\begin{question}
	Avaldage seebimullis pindpinevusteguriga \( \sigma \) ja raadiusega \( r \) olev pindpinevusest tingitud lisarõhk \( \Delta p \).
\end{question}

\section{Integraal}
Integraale kasutatakse summade, keskmisete, pindalade leidmiseks. Funktsiooni \( f(x) \) integraaliks nimetatakse pindala, mis jääb funktsiooni ja \( x \)-telje vahele. Pindala ligikaudseks leidmiseks saame jagada regiooni ristkülikuteks ja liita kokku nende pindalad. Kui valida väga õhukesed ristkülikud, siis nende pindalade summa läheneb funktsiooni integraalile. Formaalselt joone \( f(x) \)	alla jääv pindala lõigul \( [a,b] \) on
\[ \lim _ { n \rightarrow \infty } \sum _ { i = 1 } ^ { n } f \left( a+ \frac{b-a}{n}i \right) \frac{b-a}{n} = \int _ { a } ^ { b } f ( x ) dx. \]
Kui \( f(x) \) on pidev ja \( F'(x)=f(x) \), siis
\[ \int _ { a } ^ { b } f ( x ) dx=F(a)-F(b). \]
Intutiivselt võiks seda mõista, kui \( x(t) \) on keha asukoht, \( v(t)=x'(t)=\frac{dx}{dt} \) on keha kiirus, siis \( \int _ { a } ^ { b } v ( t ) dt=x(a)-x(b) \). Võrrandi vasak pool on tuleneb spidomeetri näidust ja parem pool väljendab aja t jooksul läbitud vahemaad.

Kui \( F'(x)=f(x) \), siis kehtib seos \[ \int f(x)dx=F(x)+c, \textrm{kus \( c \) on konstant}, \] mida nimetakse määramata integraaliks.

Integraalide jaoks kehtivad järgmised seosed, mille esitame tõestuseta
\[ \int c f ( x ) d x = c \int f ( x ) d x \]
\[ \int _ { a } ^ { b } f ( x ) d x = - \int _ { b } ^ { a } f ( x ) d x \]
\[ \int f ( x ) \pm g ( x ) d x = \int f ( x ) d x \pm \int g ( x ) d x \]
\[ \int u d v = u v - \int v d u \]
\[ \int _ { a } ^ { b } f ( g ( x ) ) g ^ { \prime } ( x ) d x = \int _ { g ( a ) } ^ { g ( b ) } f ( u ) d u \textrm{, kus \( u=g(x) \)}\]
\section{Ülesandeid integraali kohta}

\begin{question}[NBPhO 2016, P7][mat3][8cm]
	Osake hakkab liikuma koordinaatide algupunktist. Selle kiiruse sõltuvus ajast on antud joonisel. Milline on osakese maksimaalne nihe koordinaatide alguspunktist?
\end{question}
\begin{question}
	Leia funktsioonide \( 1/x \), \( x^n \), \( x^3(x^4+2)^5 \), \( e^{6x} \), \( xe^{6x} \) integraalid. 
\end{question}
\begin{question}
	Leidke varda inertsimoment ümber otspunkti ja ketta inertsimoment ümber keskpunkti.
\end{question}
\begin{question}
	Tuletage koonuse pindala ja ruumala valemid.
\end{question}
\begin{question}[NBPhO 2017, P9]
	Uurime kosmoselaeva, mis on homogeense toru kujuga, mille mõlemad otsad on suletud. Kosmoselaev pöörleb ümber oma massikeskme nurkkiirusega \( \omega \), et simuleerida gravitatsiooni. Pöörlemistelg on toruga risti. Kosmoselaev on täidetud õhuga, mille molaarmass on \( \mu \) ja mille rõhk pöörlemisteljel on \( p_0 \). Kosmoselaeva diameeter on palju väiksem tema pikkusest.
	\begin{subquestion}
		\item Leidke rõhk \( p \) funktsioonina kaugusest \( r \) pöörlemisteljest.
		\item Võrdluseks vaatleme (mitte pöörlevat) torni konstantses raskusväljas tugevusega \( g \), mis on täidetud sama gaasiga. Kui torni põhjas on rõhk \( p_0 \), siis mis on rõhk \( p \) funktsioonina kõrgusest \( h \) torni põhjast?
	\end{subquestion}
\end{question}
\begin{question}
	Sipelgas liigub kummiribal kiirusega \( v=1 \) cm/s. Kummiriba üks ots (see, millest hakkas sipelgas liikuma) on kinnitatud seina külge, teist ots (algselt kaugusel \( L=1 \) m seinast) tõmmatakse kiirusega \( u=1 \) m/s. Kas sipelgas jõuab kunagi kummiriba teise otsa. Kui jah, siis kui kaua palju aega see võtab?
\end{question}
\begin{question}[NBPhO 2016, P7][mat2][\columnwidth]
	Vedelat heeliumit jahutatakse seda madala rõhu all aurustades ning gaasi ära pumbates. Heeliumi aurustumissoojus on \( λ = 22 \) kJ/kg, mille võite lugeda konstantseks. Vedeliku erisoojus \( c(T) \) on kujutatud joonisel. Kui suur osa vedelikust peab aurustuma, et vähendada vedeliku temperatuuri \( T_0 = 4,1 \) K temperatuurini \( T_1 = 2,3 \) K.
\end{question}

\end{document}