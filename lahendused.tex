\documentclass[a4paper,11pt,twocolumn]{article}
\usepackage{polyglossia} % Eesti keele tugi
\setdefaultlanguage{estonian}
\usepackage{geometry}% Paigutus
\usepackage{graphicx}% Joonised
\graphicspath{ {images/} }
\usepackage{csquotes}% Eesti jutumärgid \enquote{}
\usepackage{enumitem}% Listid
\usepackage[compact]{titlesec}% Kompaktsed pealkirjad
\usepackage{siunitx}% SI ühikud
\usepackage{tikz}
\usepackage[siunitx]{circuitikz}
\usepackage[final]{microtype}
\usepackage{amsmath}
\usepackage{lmodern}
\geometry{
    paper=a4paper, % Paper size, change to letterpaper for US letter size
    top=0.5cm, % Top margin
    bottom=1cm, % Bottom margin
    left=0.5cm, % Left margin
    right=0.5cm, % Right margin
    foot=0.5cm, % Footer-margin distance
    %showframe, % Uncomment to show how the type block is set on the page
}


%\setlength{\intextsep}{0pt}

\setlength{\parindent}{0cm}% Taandridu pole
\setlength{\parskip}{1em}% Paragraafide vahed
\setlist[itemize]{topsep=0em, partopsep=0em, parsep=0em, itemsep=0.5em}% Itemize spacing

\usepackage{xparse}

% Ülesanded \begin{question}[viide][joonis][joonise suurus] (võib olla ka ainult [viide] või [joonis][joonise suurus])
\newcounter{myproblems}
\NewDocumentEnvironment{question}{o o o}
{\par \refstepcounter{myproblems} \textbf{Ülesanne \themyproblems .} \ignorespaces\IfValueT{#1}{\IfValueTF{#3}{\textbf{(#1)} \ignorespaces}{\IfNoValueT{#2}{\textbf{(#1)} \ignorespaces}}}}
{\IfValueT{#2}{\IfValueTF{#3}{\begin{figure}[h!]\includegraphics[width=#3]{#2.pdf}\centering\vspace{-1em}\end{figure}}{\begin{figure}[h!]\includegraphics[width=#2]{#1.pdf}\centering\vspace{-1em}\end{figure}}}\ignorespacesafterend}

% Alaülesanded
\newenvironment{subquestion}
{\setlength{\parskip}{0pt}\begin{enumerate}[label=\alph*), nolistsep]}
{\end{enumerate}\setlength{\parskip}{1em}\ignorespacesafterend}

% Vihjete jaoks
\newenvironment{hint}[1][Vihje]
{\setlength{\parskip}{0em} \textit{#1}: \ignorespaces}
{\setlength{\parskip}{1em}\ignorespacesafterend}

\newcommand{\pvec}[1]{\vec{#1}\mkern2mu\vphantom{#1}}% Primed vector

% Lahendused jaoks
\usepackage{hyperref}
\newenvironment{solutions}
{\begin{enumerate}[label=\textbf{\arabic*.}, wide]}
{\end{enumerate}}

% Displaystyle valemite paigutus
\makeatletter
\g@addto@macro{\normalsize}{%
    \setlength{\abovedisplayskip}{4pt}
    \setlength{\abovedisplayshortskip}{4pt}
    \setlength{\belowdisplayskip}{4pt}
    \setlength{\belowdisplayshortskip}{4pt}
    }
\makeatother

% \directlua{dofile("DetectUnderfull.lua")}
\tikzset{
    odot/.style={
        circle,
        inner sep=0pt,
        node contents={$\odot$},
        scale=1
    },
    otimes/.style={
        circle,
        inner sep=0pt,
        node contents={$\otimes$},
        scale=1
    }}


\begin{document}
{\huge \textbf{Lahendused}} \\
{Kaarel Kivisalu \hfill \today}

\section{Staatika}
\begin{solutions}
	\item \( \alpha>\arctan \mu \) \\
	\item \( \varphi=\arctan 2\mu \) \\ \url{https://www.teaduskool.ut.ee/sites/default/files/teaduskool/ainevoistlused/fyslah_2006_lahendused.pdf}, V7
	\item \( \mu=s/2h \)
	\item \( F=mg/2 \) \\ \url{https://physoly.tech/static/files/KaldaMech-121.pdf}, pr. 3
	\item a) \( F=mg\mu/\sqrt{\mu^2+1} \) b) \( F=mg \sin (\arctan \mu -\alpha) \) \\ \url{https://physoly.tech/static/files/KaldaMech-121.pdf}, pr. 4
	\item \( \mu \le \sqrt{2}-1 \) \\ \url{http://efo.fyysika.ee/yl/piirkondG/efo04v2kkl.pdf}
	\item
	\item \( a=sg/(h+b/\mu) \) \\ \url{http://efo.fyysika.ee/yl/loppvoorG/efo14v3kkl.pdf}
	\item \( F=(m_1+m_2)(\mu_1+\mu_2)g \) \\ \url{http://efo.fyysika.ee/yl/piirkondG/efo14v2kkl.pdf}
	\item \( \arctan \frac{r\mu}{(r+l)\sqrt{\mu^2+1}} \) \\ \url{https://physoly.tech/static/files/KaldaMech-121.pdf}, pr. 1
	\item \( \mu > 1 \), \( F_{min}=\frac{mg}{2} \sqrt{\frac{\mu^2+1}{\mu^2-1}} \) \\
	\url{https://dejanphysics.files.wordpress.com/2016/10/gnadig_1.pdf}
	\item \( R=\frac{L}{2}\sin^2(\frac{\alpha}{2})\tan(\frac{\alpha}{2}) \), \( \alpha < \pi/2 \)\\ \url{http://efo.fyysika.ee/yl/loppvoorG/efo17v3kkl.pdf}
	\item a) Kera on kiirem, \( \gamma=\sqrt{15/14}-1 \) b) \( \alpha_0=\arctan(3\mu) \) c) \( \alpha_m=\arctan(\frac{7}{2}\mu) \)  \\ \url{https://www.ioc.ee/~kalda/ipho/es/es2013_sol.pdf}
	\item \( F_{min}={mg}/{\sqrt{\mu^2+1}} \) \\ \url{https://www.ioc.ee/~kalda/ipho/es/es2010_sol.pdf}
	\item \( v/2 \) \\ \url{http://efo.fyysika.ee/yl/loppvoorG/efo98v3kkl.pdf}
	\item \( \alpha=\arctan\sqrt{2} \)\\ \url{http://efo.fyysika.ee/yl/loppvoorG/efo15v3kkl.pdf}
\end{solutions}

\section{Dünaamika}
\begin{solutions}
	\item Kiirused on samad, alumine kuulike jõuab enne
	\item a) \( v_1={P}/()\mu mg \) b) \( v_2={P}/[mg(\mu+0,01)] \) \\ \url{https://www.teaduskool.ut.ee/sites/default/files/teaduskool/ainevoistlused/fyslah_2014_noorem_lahendused.pdf}
	\item
	\item \url{https://web.phys.ntu.edu.tw/semi/ceos/general.files/Proofs%20of%20moments%20of%20inertia%20equations.htm}
	\item
	\item
	\item \url{https://proofwiki.org/wiki/Huygens-Steiner_Theorem}
	\item \( h=L\sqrt{M/3m} \)
	\item \( {mg}/(2M+m) \)
	\item \( m<M\cos 2\alpha \)
	\item \(\frac{mg\sin\alpha}{M+2m(1-cos\alpha)}=\frac{mg\sin\alpha}{M+4m\sin^2 \frac{\alpha}{2}} \)
	\item \( g\frac{(m_1\sin\alpha_1-m_2\sin\alpha_2)(m_1\cos\alpha_1+m_2\cos\alpha_2)}{(m_1+m_2+M)(m_1+m_2)-(m_1\cos\alpha_1+m_2\cos\alpha_2)} \)
	\item \( \cos\alpha \ge \frac{1}{3}(2+{v^2}/{gR}) \)
	\item \( g/9 \)
	\item a) \( \omega_2=\omega \) b) \( \omega=5v\cos\alpha/2R \) c) \( \mu \ge \cot \alpha \)
	\url{https://www.ioc.ee/~kalda/ipho/es/e-s-2015-sol.pdf}
	\item \( d=l\sqrt{5/2} \) \\
	\url{http://eupho2018.mipt.ru/pdf/eupho18-th-solution.pdf}
\end{solutions}

\section{Kinemaatika}
\begin{solutions}
	\item \( v_v=(s-l)/2t \), \( v_p=(s-v_vt)/t \) \\ \url{http://efo.fyysika.ee/yl/piirkondPK/efo15v2pkl.pdf}
	\item \( 4 \) km \\ \url{http://efo.fyysika.ee/yl/piirkondPK/efo14v2pkl.pdf}
	\item
\end{solutions}

\section{Matemaatika}
\section{Tuletised, diferentsiaalid ja integraalid füüsikas}
\section{Elektriahelad}
\section{Termodünaamika}
\section{Elektromagnetism}
\begin{solutions}
	\item \( F = (1 + 2\sqrt2)q^2/(8\pi\varepsilon_0 L^2) \)
	\item 15:30
	\item a) \( kq/r^2 \) b) \( \lambda/(2\pi\varepsilon_0 r) \) c) \( \sigma/(2\varepsilon_0) \) d) \( \rho r/(3\varepsilon_0) \), kui \( r<R \), muidu \( \rho R^3/(2\varepsilon_0 e r^2) \)
	\item \( E=\sigma/\varepsilon_0 \) plaaatidevahelises ruumis, mujal \( E=0 \)
	\item \( d\sqrt{2\rho e/(\varepsilon_0 m)} \)
	\item \( \varphi_0 N^{2/3} \)
	\item
	\item
	\item
    \item \(\mu_0 I/ (2\pi r)\)
	\item \( \dfrac{1}{2}\mu_0 R^2 I/(R^2+x^2)^{^3/2} \)
\end{solutions}
\end{document}
