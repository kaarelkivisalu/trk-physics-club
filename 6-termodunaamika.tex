\documentclass[a4paper,11pt,twocolumn]{article}
\usepackage{polyglossia} % Eesti keele tugi
\setdefaultlanguage{estonian}
\usepackage{geometry}% Paigutus
\usepackage{graphicx}% Joonised
\graphicspath{ {images/} }
\usepackage{csquotes}% Eesti jutumärgid \enquote{}
\usepackage{enumitem}% Listid
\usepackage[compact]{titlesec}% Kompaktsed pealkirjad
\usepackage{siunitx}% SI ühikud
\usepackage{tikz}
\usepackage[siunitx]{circuitikz}
\usepackage[final]{microtype}
\usepackage{amsmath}
\usepackage{lmodern}
\geometry{
    paper=a4paper, % Paper size, change to letterpaper for US letter size
    top=0.5cm, % Top margin
    bottom=1cm, % Bottom margin
    left=0.5cm, % Left margin
    right=0.5cm, % Right margin
    foot=0.5cm, % Footer-margin distance
    %showframe, % Uncomment to show how the type block is set on the page
}


%\setlength{\intextsep}{0pt}

\setlength{\parindent}{0cm}% Taandridu pole
\setlength{\parskip}{1em}% Paragraafide vahed
\setlist[itemize]{topsep=0em, partopsep=0em, parsep=0em, itemsep=0.5em}% Itemize spacing

\usepackage{xparse}

% Ülesanded \begin{question}[viide][joonis][joonise suurus] (võib olla ka ainult [viide] või [joonis][joonise suurus])
\newcounter{myproblems}
\NewDocumentEnvironment{question}{o o o}
{\par \refstepcounter{myproblems} \textbf{Ülesanne \themyproblems .} \ignorespaces\IfValueT{#1}{\IfValueTF{#3}{\textbf{(#1)} \ignorespaces}{\IfNoValueT{#2}{\textbf{(#1)} \ignorespaces}}}}
{\IfValueT{#2}{\IfValueTF{#3}{\begin{figure}[h!]\includegraphics[width=#3]{#2.pdf}\centering\vspace{-1em}\end{figure}}{\begin{figure}[h!]\includegraphics[width=#2]{#1.pdf}\centering\vspace{-1em}\end{figure}}}\ignorespacesafterend}

% Alaülesanded
\newenvironment{subquestion}
{\setlength{\parskip}{0pt}\begin{enumerate}[label=\alph*), nolistsep]}
{\end{enumerate}\setlength{\parskip}{1em}\ignorespacesafterend}

% Vihjete jaoks
\newenvironment{hint}[1][Vihje]
{\setlength{\parskip}{0em} \textit{#1}: \ignorespaces}
{\setlength{\parskip}{1em}\ignorespacesafterend}

\newcommand{\pvec}[1]{\vec{#1}\mkern2mu\vphantom{#1}}% Primed vector

% Lahendused jaoks
\usepackage{hyperref}
\newenvironment{solutions}
{\begin{enumerate}[label=\textbf{\arabic*.}, wide]}
{\end{enumerate}}

% Displaystyle valemite paigutus
\makeatletter
\g@addto@macro{\normalsize}{%
    \setlength{\abovedisplayskip}{4pt}
    \setlength{\abovedisplayshortskip}{4pt}
    \setlength{\belowdisplayskip}{4pt}
    \setlength{\belowdisplayshortskip}{4pt}
    }
\makeatother

% \directlua{dofile("DetectUnderfull.lua")}
\tikzset{
    odot/.style={
        circle,
        inner sep=0pt,
        node contents={$\odot$},
        scale=1
    },
    otimes/.style={
        circle,
        inner sep=0pt,
        node contents={$\otimes$},
        scale=1
    }}


\begin{document}

{\huge \textbf{Termodünaamika} \hfill \normalsize {nr 6}} \\
{Kaarel Kivisalu \hfill 28. november 2018}

\section{Soojus ja temperatuur}
\textit{Termodünaamika I seadus}: suletud süsteemi koguenergia säilib, 
\[ \Delta U=\Delta Q-\Delta W\]
kus \( \Delta U \) süsteemi siseenergia muut, \( \Delta Q \) on süsteemile antud
soojushulk, \( \Delta W \approx p\Delta V \) (\( dW=p\,dV \)) on süsteemi poolt ruumala muutmisel tehtav töö, \( p \) on gaasi rõhk ning \( \Delta V \) on ruumala muutus.

\textit{Termodünaamika II seadus}: ükskõik millisedi trikke kasutatakse (soojusmasinad, Maxwelli deemonid jne), kui välist tööd ei tehta, siis soojusvoog saab olla ainult kõrgema temperatuuriga kehalt madalama temperatuuriga kehale.

Iga keha saab iseloomustada \textit{soojusmahtuvuse} \( C \) abil,
mis näitab, kui palju soojust on vaja anda temperatuuri tõstmiseks ühe kraadi võrra: \( C = dQ/dT \). \textit{Erisoojuseks} \( c \) nimetatakse soojusmahtuvust massiühiku kohta \( c=C/m \).

Aine viimiseks ühest faasist teise on vaja sellele anda kindel soojushulk, mis on võrdeline aine massiga: \( Q=\lambda m \). Tegurit \( \lambda \) nimetatakse sulamisel \textit{sulamissoojuseks} ja aurustumisel \textit{aurustumissoojuseks}.

\textit{Stefani-Boltzmanni seadus}: \enquote{halli} keha soojuskiirguse voo tihedus on \( w=\epsilon\sigma T^4 \) (võimsus pindala kohta), kus \( \epsilon \) on neeldumistegur, \( \sigma = \SI{5,67e-8}{\watt\metre\tothe{-2}\kelvin\tothe{-4}} \).

Kui keha soojendatakse konstantsel ruumalal, siis paisumistööd ei tehta. TDIS järgi saab siis kasutada soojusmahtuvust konstantsel ruumalal, \( C_V \), siseenergia leidmiseks: \( dU=C_V\,dT \) ja \[ U=\int_{0}^{T}C_V(T')\, dT' .\]

\begin{question}[Lõppv 2018, P10]
	Vees temperatuuriga \( t_0=\SI{0}{\degreeCelsius} \) ujub jääst kuup massiga \( m_j=\SI{1,5}{\kilogram} \), mille sees on tühimik ruumalaga \( V = \SI{12}{\centi\meter\tothe{3}} \). Tühimikku valatakse hästi aeglaselt elavhõbedat temperatuuriga \( t \). Täpselt sel hetkel,	kui tühimik täitub elavhõbedaga, vajub jääst kuup põhja. Leidke tühimikku	kallatud elavhõbeda temperatuur \( t \). Jää tihedus \( \rho_j = \SI{900}{\kilogram\per\meter\cubed} \), vee	tihedus \( \rho_v = \SI{1000}{\kilogram\per\meter\cubed} \), elavhõbeda tihedus \( \rho_{Hg} = \SI{13600}{\kilogram\per\meter\cubed} \), elavhõbedaerisoojus \( c = \SI{140}{\joule\per\kilogram\per\degreeCelsius}\), jää sulamissoojus \( \lambda = \SI{330}{\kilo\joule\per\kilogram} \). Soojusvahetust väliskeskkonnaga mitte arvestada.
\end{question}
\begin{question}[Lõppv 2015, P7]
	Pliidil olevas nõus kuumutatakse \( M = \SI{0,5}{\kilogram} \) vett. Vees olev termomeeter näitab, et vee temperatuur jääb püsivalt ühtlaseks \( T_1 = \SI{80}{\degreeCelsius} \) juures. Vett kuumutatakse edasi ning vette lisatakse \( m = \SI{20}{\g} \) jää graanuleid (jää temperatuur on \( T_{j} = \SI{0}{\degreeCelsius} \)), misjärel vee temperatuur hakkab enam-vähem püsiva kiirusega langema ning aja \( t = \SI{5,0}{\minute} \) pärast on vee temperatuur langenud \( T_2 = \SI{75}{\degreeCelsius}\). Seejärel hakkab vee temperatuur tõusma ning tõuseb tagasi \( T_1 = \SI{80}{\degreeCelsius} \) juurde, kus vee temperatuur enam ei muutu. Kui suure võimsusega kütab pliit potis olevat vett, eeldades et soojuskadude võimsus on võrdeline vee ja väliskeskkonna temperatuuride vahega. Õhu temperatuur on \( T_0 = \SI{20}{\degreeCelsius} \), jää sulamissoojus \( \lambda = \SI{330}{\kilo\joule\per\kilogram} \) ja vee erisoojus \( c = \SI{4200}{\joule\per\kilogram\per\degreeCelsius}\).
\end{question}
\begin{question}
	Toas töötab radiaator võimsusega \( P=\SI{1}{\kilo\watt} \) ning toaõhu temperatuur \( t_1 = \SI{19}{\degreeCelsius} \). Õueõhu temperatuur \( t_0 = \SI{0}{\degreeCelsius} \). Tuppa tuleb inimene ning tasapisi tõuseb toaõhu temperatuur \( t_2 = \SI{20}{\degreeCelsius} \)-ni. Millise võimsusega “kütab” inimese keha?
\end{question}
\begin{question}[Lõppv 2004, G7][mat1][\columnwidth]
	Elektrikannus soojendatakse vett. Teatud hetkel pandi kannu \( T_0 = \SI{0}{\degreeCelsius}\) juures olevat jääd. Joonisel on toodud vee temperatuuri sõltuvus ajast. Kui suur oli jää mass, kui kannu võimsus \( P = \SI{1}{\kilo\watt}\). Jää sulamissoojus \( L = \SI{335}{\kilo\joule\per\kelvin}\). Toatemperatuur \( T_1 = \SI{20}{\degreeCelsius}\).
\end{question}
\begin{question}[IPhO 1996, T1]
	Soojuslikult isoleeritud metallitükki soojendatakse atmosfäärirõhul elektrivoolu abil sellisel viisil, et temal eralduv elektriline võimsus on ajas konstantne ja võrdne \( P \)-ga. Selle tagajärjel kasvab metalli absoluutne temperatuur ajas seaduse
	\[ T(t) = T_0[1 + a(t − t_0)]^{1/4} \]
	kohaselt, kus \( a \), \( t_0 \) ja \( T_0 \) on konstandid. Määrata metalli soojusmahtuvus \( C(T) \) (antud katses kasutatud temperatuurivahemiku jaoks sõltub see temperatuurist).
\end{question}
\begin{question}
	Tehiskaaslase kere on kera läbimõõduga \SI{1}{\meter}, mille kõikides punktides on ühesugune temperatuur. Tehiskaaslane asub kosmoses Maa läheduses, aga mitte tema varjus. Päikese pinna kui absoluutselt musta keha pinna temperatuur \( T_\odot = \SI{6000}{\kelvin} \), Päikese raadius \( R_\odot = \SI{6,96e8}{\meter} \), Maa ja Päikese vaheline kaugus \( L = \SI{1,5e11}{m} \). Leia tehiskaaslase temperatuur eeldusel, et see on kaetud ideaalse halli värviga (neeldumistegur ei sõltu kiirguse lainepikkusest).
\end{question}
\begin{question}[IPhO 1996, T1]
	Absoluutselt musta kuuma tasapinda hoitakse konstantsel temperatuuril \( T_h \). Temaga on paralleelne teine absoluutselt must tasapind, mida hoitakse madalamal konstantsel temperatuuril \( T_l \). Plaatide vahel on vaakuum. Selleks, et vähendada kiirgusest tingitud soojusvoogu soojalt tasapinnalt külmale tasapinnale kasutatakse ekraani, mis koosneb \( N \) üksteisega paralleelsest ja üksteisest soojuslikult isoleeritud absoluutselt mustast plaadist. See ekraan asetatakse kuuma ja külma plaadi vahele, paralleelsena nii sooja kui külma tasapinnaga. Teatud aja pärast tekib süsteemis statsionaarne olek. Millise koefitsiendi \( x \) võrra kahandab ekraan soojusvoogu külma ja sooja tasapinna vahel? Pindade lõplikest mõõtmetest
	tingitud ääreefekte mitte arvestada.
\end{question}

\section{Molekulaarkineetiline teooria}
Heaks mudeliks, mis kirjeldab reaalsust üsna tõetruult, on \textit{ideaalse gaasi mudel}. Ideaalse gaasi puhul loetakse, et molekulid on nagu tühiselt väikeste mõõtmetega absoluutselt elastsed kuulikesed (või vedrude abil ühendatud kuulikeste süsteemid), mis liiguvad juhuslikult, vastu seinu ja üksteist põrgates; molekulid mõjutavad üksteist vaid siis, kui nad on vahetus kontaktis ning nendevaheline kaugmõju (nt elektrostaatiline) puudub.

Leiame rõhu avaldise. Eeldame, et kõigi molekulide kiirus \( x \)-teljes on \( v_x \) ning \( x \)-teljega risti on sein. Rõhk on jõud pindalaühiku kohta, \( p=F/A \); jõud on seinale ajaühikus üle antud impulss, \( F=2Nmv_x/t \), kus \( N \) on aja \( t \) jooksul seinaga põrkuvate molekulide arv. Kordaja 2 tuleb sellest, et molekulide seinast eemaldumise keskmine kiirus pärast põrget on sama suur kui lähenemise kiirus. Aja \( t \) jooksul jõuavad seinani need molekulid, mis on seinaäärses kihis paksusega \( v_x t \) ja liiguvad seina poole. Selle kihi ruumala on \( V=Av_xt \) ja molekulide arv on \( N=\frac{1}{2}nV = \frac{1}{2}nAv_xt \) (kordaja \( \frac{1}{2} \) tuleb sellest, et ainult pooled molekulid liiguvad seina poole). Järelikult on jõud \( F=2Nmv_x/t=nmv_{x}^2A \). 

Reaalsuses on molekulide kiirused erinevad ja tuleb vastust keskmistada: \( F=nm\langle v_{x}^2 \rangle A \). Saame, et rõhk, \( p=F/A \), on \( p=nm\langle v_{x}^2 \rangle. \) Kui molekulide kogukiirus on \( v \), siis kehtib avaldis \( v^2=v_x^2+v_y^2+v_z^2 \), järelikult \( \langle v^2 \rangle = 3 \langle v_x^2 \rangle \). 

Maxwelli jaotust kasutades vaame avaldada keskmise kiiruse integraalina: \( \langle v_x^2 \rangle = \int v_x^2 f(v_x) \, dv_x =k_B T/m \). Asendades selle rõhu võrrandisse saame, et \( p=nm\langle v_{x}^2 \rangle=nk_B T. \) 

Asendades \( n=N/V=\frac{m}{\mu}N_A/V \) (\( N_A \) on Avogadro arv, \( \mu \) on gaasi molaarmass, \( m \) on gaasi kogumass, \( V \) on gaasi ruumala) saame, et \( pV=\frac{m}{\mu}N_A k_B T \equiv \frac{m}{\mu}RT \), kus \( R \) on universaalne gaasikonstant. Seda seadust tuntakse \textit{ideaalse gaasi olekuvõrrandina.} Mõnikord on see valem mugav esitada ka gaasi tiheduse \( \rho \) abil: \( p\mu=\rho RT \).

Ideaalse gaasi puhul kehtib olekuvõrrand 
\[ pV=\nu RT, \] 
mille ekvivalentsed kuhud on 
\[ p=n k_B T \ \textrm{ja}\ p\mu=\rho RT. \]

\begin{question}
	Vaakumis kaalutuse tingimustes asuva silindrilise klaasi põhjas on tahke aine molaarmassiga \( \mu \). Tasapisi see aine sublimeerub (aurab tahkelt pinnalt gaasina) ning tõukab seeläbi klaasi avausele vastassuunas. Hinnake, millise kiiruse	omandab klaas, kui klaasi mass on \( M \) ja seal sees olnud aine mass \( m \ll M \); aine temperatuur on \( T \) ning soojuskiirgusest tingitud jahtumisega mitte arvestada.
\end{question}
\begin{question}
	Looduslik uraan koosneb põhiliselt kahest isotoobist, U\( ^{238} \) ja U\( ^{235} \), kusjuures viimast on umbes \SI{0,7}{\percent}. Uraani rikastamiseks (U\( ^{235} \) sisalduse tõstmiseks) kasutatakse mitme-etapilist protsessi, kus igal etapil lastakse gaasilises olekus oleval uraani-ühendil UF\( _{6} \) tungida läbi poorse vaheseina. Poorset vaheseina võib vaadelda kui õhukest kilet, milles on mikroskoopilised augud (väiksemad molekulide vaba tee pikkusest, kuid suuremad molekuli mõõtmetest). Mitme etapi tulemusel suureneb U\( ^{235} \) sisaldus \SI{1,4}{\percent}-ni? Fluori molaarmass on \SI{19}{\g\per\mol}.
\end{question}

Üheaatomilise gaasi siseenergia \( U \) (st molekulide soojusliikumise energia) arvutamiseks võime kasutada eespoolsaadud tulemust \( m\langle v^2_x\rangle = kT \). Tõepoolest, üheaatomilise molekuli mõõtmed on nii väikesed, et normaaltingimustel ta ei saa pöörelda. Seega on koguenergia molekulide arvu ja üksiku molekuli kineetilise energia keskväärtuse korrutis: \( U=N\frac{m}{2} (\langle v^2_x\rangle + \langle v^2_y\rangle + \langle v^2_z\rangle).\) Kõikide suundade võrdväärsuse tõttu \( \langle v^2_x\rangle = \langle v^2_y\rangle = \langle v^2_z\rangle \); arvestades, et \( N=\nu N_A \) ja \( N_A k_B=R \), saame lõpptulemuse
\[ U=\frac{3}{2} \nu RT.\]
Kahe- ja enama-aatomiliste molekulide puhul tuleb kulgliikumise energiale liita veel pöördliikumise energia \( U_p =\frac{1}{2} N(I_x\omega_x^2 + I_y\omega_y^2 + I_z\omega_z^2)=\frac{3}{2} \nu RT \), kus \( I_x \), \( I_y \) ja \( I_z \) tähistavad inertsimomente erinevate telgede ümber pöörlemisel ning \( \omega_x \), \( \omega_y \) ja \( \omega_z \) vastavaid nurkkiirusi. Siiski, nii nagu ühe-aatomiline molekul ei saa üldse pöörelda, ei saa ka kahe-aatomiline (ja enama-aatomiline, kuid lineaarne) molekul pöörelda ümber kahte aatomit ühendava telje. Seetõttu jääb sellistel molekulidel üks liidetav (kolmest) ära ning \( U_p=\nu RT .\) 

Paneme tähele, et iga nn vabadusaste (liikumine \( x \)-telje suunas, pöörlemine \( x \)-telje ümber, liikumine \( y \)-telje sihis jne) annab ühesuguse panuse siseenergiasse, \( \frac{1}{2}\nu RT \) (või kui rääkida üksiku molekuli keskmisest energiast, siis \( \frac{1}{2}kT \)). Seetõttu on mugav kasutada vabadusastmete arvu \( i \) gaasi molekulide iseloomustamiseks ning kirjutada siseenergia avaldis kujul
\[ U=\frac{i}{2} \nu RT.\]
Ühe-aatomilise gaasi jaoks \( i = 3 \), kahe-aatomilise (ja lineaarsete molekulidega enama-aatomilise) gaasi jaoks \( i = 5 \) ning ülejäänutel puhkude \( i = 6 \). Gaaside segu puhul (nt õhk) võib keskmine vabadusastmete arv osutuda murdarvuks. Kasutades seda avaldist on meil lihtne leida ideaalse gaasi soojusmahtuvus konstantse ruumala juures. Kui ruumala ei muutu, siis rõhumisjõud tööd ei tee ning järelikult läheb kogu soojus siseenergia suurendamiseks, st \( C_V = dU/dT = \frac{i}{2} \nu R \). Meeldejätmiseks on see mugav esitada molaarse soojusmahtuvusena \( c_V = C_V /\nu \):
\[ c_V = \frac{i}{2} R. \]
Teisest küljest, konstantse rõhu puhul ruumala kasvab vastavalt eelpooltuletatud olekuvõrrandile \( p\, dV = νR\, dT \). Seetõttu teeb gaas välise rõhu vastu tööd \( dW = p\, dV = \nu R\, dT \) ning TDIS põhjal \( dQ = dU + W = \frac{i+2}{2} \nu R\, dT \), st
\[ c_p=\frac{i+2}{2}R.\]
Asendades \( c_V \) gaasi siseenergia avaldisse saame \( U=c_V \nu T \).

\begin{question}[Lõppv 2006, G6]
	Vesinikuga täidetud aerostaat (venimatust nahast õhupall) ruumalaga \( V_0 \) hõljub stabiilsel kõrgusel, kus on muutumatu õhurõhk \( p \) ja õhutemperatuur \( T_{ohk} \). Päikesekiirguse käes soojenes aerostaat temperatuurini \( T_1 \). Selle tulemusel väljus aerostaadist osa gaasi ventiili kaudu, mis laseb gaasi välja, kuid mitte sisse ja ei lase rõhul aerostaadis tõusta kõrgemaks välisrõhust. Seejärel varjus päike pilve taha ja aerostaat jahtus taas ümbritseva õhu temperatuurini, mistõttu tema ruumala vähenes. Kui palju ballasti tuleb üle parda heita, et aerostaat säilitaks algse kõrguse? Vesinik ja õhk lugeda ideaalseteks gaasideks, mille molaarmassid \( \mu _{H_{2}} \) ja \( \mu_{ohk} \) on teada. Eeldada, et aerostaadi kõrgus soojenemise-jahtumise käigus ei jõudnud praktiliselt muutuda.
\end{question}
\begin{question}
	Vaatleme ballooni, mis on algselt õhust tühjaks pumbatud. Ballooni ventiil keeratakse natuke lahti ning õhk voolab ballooni sisse. Millise temperatuuri omandab õhk balloonis peale seda, kui õhu sissevool on lakanud? Toatemperatuur on \( T \), ballooni seinte soojusjuhtivuse ja soojusmahtuvusega mitte arvestada.
\end{question}

\section{Adiabaatiline protsess}
Adiabaatiliseks nimetatakse sellist protsessi, mis on
nii kiire, et soojusvahetust ei toimu, aga nii aeglane, et tema kulgemise karakteerne aeg on mõnevõrra väiksem (praktikas piisab paarikordset erinevusest), kui süsteemi omavõnkesagedus.

Gaaside puhul tähendab seda, et ei eraldu ega neeldu soojust: \( \Delta Q=0 \).

Olümpiaadidel pole üldjuhul vaja adiabaadi seaduse tuletust teada, piisab adabaadi seaduse teadmisest. Rakendame TDIS ideaalse gaasiga toimuvale adiabaatilisele protsessile: \( dU=-p\, dV \), kus \( dU=C_V \, dT \), st \[ C_V  \, dT=-p\, dV .\]
Tahame leida rõhu ja ruumala vahelist seost, seega on vaja elimineerida temperatuur. Selleks kasutame ideaalse gaasi seadust \( \nu RT=pV \), järelikult \( \nu R\,dT=p\, dV+V\, dp. \) Kombineerides võrrandeid saame \( p\,dV(c_V+R)+c_V V\, dp=0. \) Märkame, et \( c_V+R=c_p \), ja toome sisse adabaadinäitaja mõiste, 
\[ \gamma=c_p/c_V. \]
Selle abil saame viimase võrrandi ümber kirjutada kujul
\[ \gamma \frac{dV}{V}+\frac{dp}{p}=0. \]
Kui me integreerime seda võrrandit, siis esimene liige on \( \int \gamma \frac{dV}{V}=\gamma \ln V=\ln V^\gamma \), ja teine liige on \( \int \frac{dp}{p}=\ln p \), seega \( \ln V^\gamma +\ln p=\ln pV^\gamma=Const \) ja järelikult ka 
\[ pV^\gamma=Const. \]
Harilikult on õhumassid atmosfääris tugevas liikumises, nii et nad liiguvad pidevalt üles-alla. Et kõrgemal on rõhk väiksem, kui all, siis jahtub gaas üles liikudes heas lähenduses adiabaatilise paisumise tõttu (õhumasside suurte mõõtmete tõttu on
soojusjuhtivus hästi aeglane).

\begin{question}
	Milline on õhu temperatuur \(  H = \SI{200}{\m} \) kõrguse mäe otsas, kui jalamil on sooja \( t_0 = \SI{20}{\degreeCelsius} \)? Õhumasside liikumisel üles- või allapoole võib õhu paisumise ja kokkusurumise lugeda adiabaatiliseks protsessiks. Õhurõhk maapinnal \( p_0 = \SI{10e5}{\pascal} \), õhu tihedus \( \rho = \SI{1,29}{\kg\per\m\cubed} \), adiabaadinäitaja \( \gamma = \num{1,4} \), vabalangemise kiirendus \( g = \SI{9,81}{\m\per\s\squared}  \).
\end{question}
\begin{question}[Lõppv 2018, G8]
	Ilusa päikeselise ilma korral on harilikult tegemist nn adiabaatilise atmosfääriga. See tähendab, et õhumassid on pidevas üles-alla liikumises. Kerkides õhk paisub ja jahtub adiabaatiliselt; pideva segunemise tõttu on kerkiva õhumassi temperatuur võrdne seda antud kõrgusel ümbritsevate õhumasside temperatuuriga. On võimalik näidata, et sellisel juhul kahaneb temperatuur lineaarselt kõrgusega, \( T=T_0-\frac{\gamma-1}{\gamma}\frac{\mu g h}{R} \), kus õhu adiabaadinäitaja \( \gamma = \num{1,4} \), õhu keskmine molaarmass \( \mu = \SI{29}{\g\per\mol} \), vabalangemise kiirendus \( g = \SI{9,81}{\m\per\s\squared}  \), gaasikonstant \( R = \SI{8,31}{\J\per\mol\per\K}\) ja kõrgus maapinnast on \( h \); õhutemperatuur maapinnal \( T_0 = \SI{293}{\K} \). Venimatust kuid vabalt painduvast nahast valmistatud õhupall mahutab maksimaalselt ruumala \( V_0 \) jagu gaasi; see täidetakse sellise koguse heeliumiga, mis võtab maapinnal enda alla ruumala \( V_0/2 \). Õhupall lastakse lahti ja see hakkab aeglaselt kerkima; lugeda, et õhupallis oleva heeliumi temperatuur on kogu aeg võrdne ümbritseva õhu temperatuuriga. Hinnake, millisel kõrgusel \( h_1 \) on õhupalli tõstejõud \SI{1}{\percent} võrra väiksem kui maapinnal. Teilt oodatakse sellist hinnangut kõrgusele, mille suhteline viga pole suurem ühest kümnendikust, kusjuures vea piisavat väiksust pole vaja tõestada
\end{question}
\begin{question}
	Tuletage eelmises ülesandes kasutatud seos \( T=T_0-\frac{\gamma-1}{\gamma}\frac{\mu g h}{R} \).
\end{question}
\begin{question}[IPhO 1987][ter1][8cm]
	Üle mäeaheliku voolab adiabaatiliselt õhk, mis on niiske (vt joonis). Meteoroloogiajaamad \( M_0 \) ja \( M_3 \) mõõdavad õhurõhuks \( p_0 = \SI{100}{\kilo\pascal} \), jaam \( M_2 \) aga \( p_2 = \SI{70}{\kilo\pascal} \). Õhu temperatuur punktis \( M_0 \) on \( t_0 = \SI{20}{\degreeCelsius}\). Õhumassi kerkides algab rõhu \( p_1 = \SI{84,5}{\kilo\pascal}\) juures pilvede moodustumine. Edasi kulub õhumassil mäeharjani (jaamani \( M_2 \)) jõudmiseks aega \SI{1500}{\s} ning lõppkokkuvõttes sadeneb igast kilogrammist õhust sademete (vihma) näol välja \( m = \SI{2,45}{\g}\) vett. Iga ruutmeetri kohal on sellist niisket õhku, kus toimub vee kondenseerumine kokku \SI{2000}{\kg}.
	\begin{subquestion}
		\item Leidke õhu temperatuur \( T_1 \) pilvede alumise piiri juures.
		\item Millisel kõrgusel \( h_1 \) jaamast \( M_0 \) asub pilvede alumine piir, kui eeldada, et õhu tihedus väheneb kõrguse kasvades lineaarselt?
		\item Milline on temperatuur \( T_2 \) mäe harja juures?
		\item Milline sademetehulk pindalaühiku kohta sajab maha kolme tunni jooksul? Vastus andke veekihi paksusena millimeetrites. Kondenseerumistingimused lugeda ühesuguseks üle kogu mäenõlva, punktist \( M_1 \) kuni punktini \( M_2 \).
		\item Milline on õhutemperatuur teisel pool mäeahelikku jaamas \( M_3 \)? Mille poolest erineb õhu olek jaamas \( M_3 \) tema olekust jaamas \( M_0 \)?
	\end{subquestion}
	\begin{hint}[Juhtnööre ja arvandmeid]
		Õhku võib vaadelda ideaalse gaasina; veeauru mõju õhu tihedusele ja erisoojusele ning aurustumissoojuse sõltuvust temperatuurist ignoreerida. Temperatuurid leida täpsusega \SI{1}{\K}, pilvede alumise piiri kõrgus täpsusega \SI{10}{\m}, sademete hulk täpsusega \SI{0,1}{\mm}. Õhu erisoojus meid huvitavas temperatuurivahemikus \( c_p = \SI{1005}{\J\per\kg\K} \); õhu tihedus jaama \( M_0 \) juures rõhul \( p_0 \) ja temperatuuril \( t_0 \) on \( \rho_0 = \SI{1,189}{\kg\per\m\cubed} \); vee aurustumissoojus pilvede piirkonnas on \( q_V = \SI{2500}{\kJ\per\kg} \); adiabaadinäitaja \( \gamma = c_p/c_V = \num{1,4} \); vabalangemiskiirendus \( g = \SI{9,81}{\m\per\s\squared} \).
	\end{hint}
\end{question}

Statsionaarsete gaasivoolude kohta käivates ülesannetes saab kasutada kahte jäävusseadust. 

Esiteks, mööda voolujooni
\[ \frac{v^2}{2}+gh+c_pT=Const, \]
kus \( c_p=C_p/m \) on gaasi erisoojus konstantsel rõhul (\( C_p \) on soojusmahtuvus konstantsel rõhul, \( m \) - gaasi mass). See võrrand tuleneb energia jäävusest. Alternatiivselt, kui voolu kiirus on palju väiksem heli kiirusest (võrdväärne väitega, et tiheduse muutused on voolujoontel väiksed, \( \frac{\Delta \rho}{\rho}\ll 1 \)), originaalset Brenulli võrrandit
\[ \frac{\rho v^2}{2}+\rho hg+p=const \]
saab samuti kasutada.

Teiseks,
\[ \rho v A=const ,\]
kus \( A \) on voolujoonte moodustatud fiktiivse toru ristlõikepindala; see tuleneb massi jäävusest, \( \rho v \) on võrdne massi voo tihedusega.

\begin{question}[IPhO 2012, T1B][ter2][\columnwidth]
	Ülesande selle osa jaoks võib kasuks tulla järgmine informatsioon. Vedeliku või gaasi voolu korral kehtib mööda voolujoont seos \( p+\rho g h+\frac{1}{2}\rho v^2=const \), eeldusel et kiirus \( v \) on palju väiksem kui heli kiirus. Siin kasutatakse tähistusi \( \rho \) -- tihedus, \( h \) -- kõrgus, \( g \) -- raskuskiirendus, \( p \) -- hüdrostaatiline rõhk. Voolujooned on defineeritud kui gaasiosaskeste trajektoorid (eeldusel et need on ajas muutumatud). Avaldist \( \frac{1}{2}\rho v^2 \) nimetatakse dünaamiliseks rõhuks.\\
	Alltoodud joonisel on kujutatud lennukitiiva läbilõige koos ümbritsea õhu liikumise voolujoontega, vaadatuna tiiva taustsüsteemis. Eeldage, et (a) õhuvool on täiesti kahedimensionaalne (s.t õhu kiirusvektorid on tervenisti joonise tasandis); (b) voolujoonte kuju ei sõltu lennuki kiirusest; (c) ei ole tuult; (d) dünaamiline rõhk on palju väiksem kui atmosfäärirõhk \( p_0=\SI{1,1e5}{\Pa} \). Võite joonlaua abil teha jooniselt vajalikke nõõtmisi.
	\begin{subquestion}
		\item Kui lennuki kiirus maa suhtes on \( v_0=\SI{100}{\m\per\s} \), siis mis on õhu kiirus \( v_p \) maapinna suhtes punktis \( P \) (märgitud joonisel)?
		\item Kõrge suhtelise õhuniiskuse puhul, kui lennuki kiirus maa suhtes uletab kriitilise väärtuse \( v_{crit} \), tekib tiiva taga veepiiskade juga. Piisad tekivad teatud punktis \( Q \). Tähistage punkt \( Q \) joonisel. Seletage kvalitatiivselt (kasutades valemeid ja võimalikult vähe teksti), kuidas te selle punkti asukoha leidsite.
		\item Hinnake kriitilist kiirust \( v_{crit} \) järgmiseid andmeid kasutades: algse häirimata õhu suhteline niiskus \( r=\SI{90}{\percent} \), õhu erisoojus konstantsel rõhul \( c_p=\SI{1,00e3}{\J\per\kg\per\K} \), küllastunud veeauru rõhk: \( p_{sa}=\SI{2,31}{\kPa} \) algse häirimata õhu temperatuuril \( T_a=\SI{293}{\K} \) ja \( p_{sb}=\SI{2,46}{\kPa} \) algse häirimata õhu temperatuuril \( T_b=\SI{294}{\K} \). Sõltuvalt kasutatud lähendusest võib lisaks vaja minna õhu erisoojust konstantsel ruumalal \( c_V=\SI{0,717e3}{\J\per\kg\per\K} \). \hfill
	\end{subquestion}
	\setlength{\parskip}{0em}
	Suhteline õhuniiskus on defineeritud aururõhu ja küllastunud auru rõhu suhtena. Küllastunud auru rõhk on defineeritud kui aururõhk, mille puhul aur on tasakaalus vedelikuga.
	\setlength{\parskip}{1em}
\end{question}

\section{Vedeliku ja auru eralduspind}
Osakese faasisiirdeks vedelast faasist gaasilisse on sellel vaja omandada energia \( U_0 \). Molekuli jaoks on tõenäosus, et ta lisaenergia on suurem kui \( U_0 \) võrdeline \( e^{-U_0/k_BT} \)-ga; \textit{aurumiskiirus} on võrdeline selle tõenäosusega ja järelikult, suureneb väga kiiresti temperatuuri kasvuga. \textit{Kondenseerumiskiirus} on temperatuurile vähem tundlik (võrdeline molekulide kiirusega, st \( \propto \sqrt{T} \)) ja sõltub peamiselt auru molekulide kontsentratsioonist gaasilises olekus.

Auru, mille kontsentratsioon on selline, et selle aurumiskiirus ja kondenseerumiskiirus on võrdsed antud temperatuuril nimetatakse \textit{küllastunud auruks}.

Kehtib Daltoni seadus, mis väidab, et gaaside segu rõhk on osarõhkude summa: \( p=p_A+p_B+... \), kus \( p \) on kogurõhk, \( p_A \) -- rõhk \( A \) tüüpi molekulide tõttu, \( p_B \) -- rõhk \( B \) tüüpi molekulide tõttu jne.

\textit{Suhteliseks õhuniiskuseks} nimetatakse aururõhu ja küllastunud auru rõhu suhet antud gaasi temperatuuril.

Kahe vedeliku piirpinnal algab keemine märksa madalamal temperatuuril, kui kummaski vedelikus eraldi. Keemine hakkab tingimusel kui küllastunud auru rõhk on suurem atmosfääri rõhust: \( p_s(T)>p_{atm} \). Kahe vedeliku piirpinnal kehtib keemise jaoks tingimus \( p_{1s}+p_{2s}>p_{atm} \).
\begin{question}
	Vaakumis ja kaalutus olekus, silindrilise anuma põhjas asub kiht tahket ainet. See sublimeerub (aurustub tahkest faasist gaasilisse) aeglaselt ja lükkab selle käigus anumat vastassuunas. Anuma mass on \( M \) ja aine algne mass \( m \ll M \). Anuma ristlõike pindala on \( A \) ja küllasutunud auru rõhk temperatuuril \( T \) on \( p_0 \). Milline on anuma kiirendus, kui 
	\begin{subquestion}
		\item molekulide keskmine vaba tee pikkus \( \lambda \) küllastunud aururõhul on palju väiksem anuma pikkusest;
		\item palju suurem sellest.
	\end{subquestion}
\end{question}
\begin{question}[ter3][8 cm]
	Geisreid võib vaadelda kui suuri maa-aluseid reservuaare, mis on täidetud põhjaveega ja mida kuumutavad reservuaari kuumad seinad. Reservuaar on ühendatud maapinnaga kitsa ja sügava kanali abil, mis on geisri puhkeolekus täidetud peaaegu servani veega. Lugeda, et geiser hakkab purskama siis, kui vesi reservuaaris läheb keema; puhkeperioodi jooksul vesi maapinnale viivas kanalis ei kuumene; purske ajal on kanal täidetud ainult veeauruga; reservuaari sees seguneb vesi hästi, nii et vee temperatuur on seal kõikjal enam-vähem üks ja sama; reservuaari maht on hulga suurem kanali mahust; põhjavee juurdevool läbi reservuaari seinte on piisavalt kiire, et täita pursete vahelise aja jooksul reservuaar ja kanal kuni kanali ülemise servani. Olgu maapinnale viiva kanali otste kõrguste vahe \( h = \SI{90}{\m} \). Vee aurustumissoojus \( \lambda = \SI{2,26e6}{\J\per\kg}\) ja tema erisoojus \( c = \SI{4,2e3}{\J\per\kg\per\K} \). Vee küllastunud auru rõhu sõltuvus temperatuurist on toodud graafikul. Leidke, millise osa oma veest kaotab reservuaar purske jooksul. 
\end{question}
\begin{question}[IPhO 1989, T1][ter4][7cm]
	Vedelikud \( A \) ja \( B \) omavahel ei segune. Küllastunud auru rõhk \( p_i (i = A, B) \) nende kohal vastab piisava täpsusega valemile \( ln(p_i/p_0) = a_i/T + b_i \), kus \( p_0 \) on atmosfääri normaalrõhk, \( T \) on auru absoluutne temperatuur ning \( a_i \) ja \( b_i \) on vedeliku omadustest sõltuvad konstandid. Suhte \( p_i/p_0, i = A, B \) tegelikud väärtused temperatuuride \SI{40}{\degreeCelsius} ja \SI{90}{\degreeCelsius} jaoks on esitatud tabelis.
	\begin{table}[h!]
		\vspace{-1em}
		\centering
		\begin{tabular}{ccc}
			\( T(\si{\degreeCelsius}) \) & \( p_A/p_0 \) & \( p_B/p_0 \) \\ 
			\num{40} & \num{0,284} & \num{0,07278} \\ 
			\num{90} & \num{1,476} & \num{0,6918} \\ 
		\end{tabular}
		\vspace{-1em} 
	\end{table}
	\begin{subquestion}
		\item Leidke vedelike \( A \) ja \( B \) keemistemperatuurid.
		\item Vedelikud \( A \) ja \( B \) valati nõusse, kus nad kihistusid nii, nagu näidatud joonisel. Vedelikku \( B \) katab õhuke mitteaurustuva vedeliku C kiht, mis ei lahustu ei \( A \)-s ega \( B \)-s. Vedelike \( A \) ja \( B \) molaarmassid gaasilises olekus suhtuvad nagu \( g = M_A/M_B = \num{8} \). Algselt oli mõlemat vedelikku ühepalju, \( m = \SI{100}{\g} \) kumbagi. Vedelikusamba kõrgus ja nende tihedus on sellised, et kõikjal anumas võib rõhu lugeda võrdseks välisrõhuga \( p_0 \). Seda segu hakatakse aeglaselt ja ühtlaselt kuumutama. Skemaatiliselt on vedeliku temperatuuri \( t \) sõltuvus ajast \( t \) esitatud graafikul. Leidke temperatuurid \( t_1 \) ja \( t_2 \) ühe kraadilise täpsusega. Millised on vedelike massid ajahetkel \( t_1 \)?\\
		\begin{hint}[Märkus]
			Eeldatakse, et vaadeldavate vedelike aurud alluvad piisava täpsusega Daltoni seadusele ning kuni küllastunud auru rõhuni võib neid lugeda ideaalseteks gaasideks.
		\end{hint}
	\end{subquestion}
\end{question}

\section{Pindpinevus}
Vedelikus olevad molekulid omandavad keemiliste sidemete tõttu negatiivse potentsiaalse energia aurufaasis olevate molekulide suhtes. Äärmises pinnakihis on igal molekulil lähinaabreid ainult ühel pool ning seetõttu on nende negatiivne potentsiaalne energia natuke väiksem. Teisisõnu, vedeliku pinnal asuvatel molekulidel on teatav positiivne potentsiaalne energia vedeliku sisemuses asuvate molekulide suhtes. Seda energiat võib vaadelda kui vedeliku vaba pinnaga seonduvat energiat, mis loomulikult on võrdeline vaba pinna pindalaga, \( U = S\sigma \). Võrdetegurit \( \sigma \) nimetatakse pindpinevusteguriks.

Kui vaadelda mõtteliselt joont pikkusega \( l \) vedeliku pinnal, siis kaks poolt, milleks see joon vedeliku pinna jagab, tõmbuvad jõuga \( F = \sigma l \). Tõepoolest, kui lasta vedeliku pinnal
laieneda risti mõttelise joonega \( dx \) võrra, siis
pindala suureneb \( dS = l\,dx \) võrra ning potentsiaalne energia
\( dU = \sigma l\,dx \) võrra. Selle jaoks on vaja teha tööd, st peab toimima jõud \( F \), mis rahuldab võrdust \( F\,dx = dU \), st \( F = \sigma l \).

Meniski, kapillaaris oleva veesamba, toru otsas rippuva tilga jms massi saab leida tasakaalutingimusest: pindpinevusjõud tasakaalustab raskusjõu.

Teatud puhkudel on pindpinevusülesannetes võimalik leida tasakaaluasend summaarse energia (pinna-energia koos ülejäänud potentsiaalse energiaga) miinimumina.

Tihti on kasulik kasutada jõudude tasakaalu tingimust fiktiivselt eraldatud vedeliku osade jaoks gravitatsioonijõu, pindpinevusjõu ja hüdrostaatilisest rõhust tingitud jõu suhtes.

Kõver pindpinev pind põhjustab rõhkude erinevust (ühe ja teise ruumipoole vahel). Sfäärilise pinna puhul \( \Delta p = 2\sigma/R \) ja silindrilise pinna puhul \( \Delta p = \sigma/R \), kus \( R \) on kõverusraadius.

Pindpinevusülesannetes figureerivadki sageli kiled (sh seebimullid). Sellisel juhul ei tohi unustada, et kilel on kaks külge ning mõlemal küljel on oma pindpinevusenergia, mis tekitab valemitesse teguri \enquote{2}.

\begin{question}
	Silindrilise mensuuri mahtuvus on \( V = \SI{100}{\ml} \) ja kõrgus \( h = \SI{20}{\cm} \). Mensuurile kantud skaalal on jooned tõmmatud iga \SI{1}{\ml} järel. Oletame, et kasutakse mensuuri vee ruumala mõõtmiseks ja loetakse vee nivooks vee vaba pinna madalaima punkti kõrguse. Milline viga tuleneb sellest asjaolust, et vee pind on tegelikult kõver? Vee pindpinevustegur \( \sigma = \SI{0,073}{\N\per\m} \) ja ta märgab täielikult mensuuri seinu.
\end{question}
\begin{question}
	Süstalt hoitakse vertikaalselt nõelaga allapoole; seda vajutatakse aeglaselt nii, et tilgad kukuvad nõela tipust (nõela tipp on lame, st et on lõigatud risti nõela teljega). Pindpinevustegur on \( \sigma \), tihedus on \( \rho \), raskuskiirendus on \( g \), nõela sisediameeter on \( d \) (\( d \ll \sqrt{\sigma/\rho g} \)). Leia langeva tilga mass \( m \) (andke oma vastusele parand võttes arvesse tilga siserõhust tingutud jõudu).
\end{question}

\begin{question}[E-S 2009, P6][ter5][5cm]
	Lord Rayleigh pidas 1891. aastal loengu füüsikaliste protsesside fotografeerimisest, muuhulgas esitles ta ka fotot rõngale toetunud purunevast seebikilest (pilt lisatud). Välklambi asemel kasutas ta valgustamiseks lühiajalist elektrisädet (elektrisädemel põhinevad tegelikult ka tänapäevased välklambil). Hinda, kui täpselt pidi Lord Rayleigh selle sädeme tekkimise ajastama - teisisõnu, hinnake seebikile purunemisele kuluvat aega. Seebikile paksus olgu \( h = \SI{1}{\micro\m} \), rõnga diameeter \( D = \SI{10}{\cm} \) ning seebivee pindpinevustegur \( \sigma = \SI{0.025}{\N\per\m} \).\\
	\begin{hint}
		Kasutage mudelit, mille kohaselt juba purunenud seebikile osa koondub ühtseks frondiks ning hakkab ühtselt liikuma purunemata osa poole.
	\end{hint}
\end{question}

\begin{question}[200MPPP, P67][ter6][\columnwidth]
	Süüa tehes võib märgata, et viiner lõheneb keetmisel. See toimub alati \enquote{pikkupidi}, mitte kunagi aga \enquote{ristipidi}. Mis on selle põhjus? Kui oleks võimalik luua toroidikujuline viiner, siis kust ja millises suunas see lõheneks. Eeldage mõlemal juhul, et viinerinahk on homogeenne.
\end{question}
	
\section{Ülesandeid}

\begin{question}
	Ulmeromaanis kirjeldatakse järgmist olukorda. Kosmoselaeval toimub avarii ning astronaut osutub olevaks avakosmoses, kaugusel \( L = \SI{100}{\m} \) kosmoselaevast. Tal on juhtumisi käes klaas tahkestunud veega (jääga), ja ta kasutab jää sublimeerumisel (aurustumisel) tekkivat reaktiivjõudu selleks, et naasta kosmoselaevale. Hinnake, kui realistlik see projekt. Võite eeldada, et sublimeerumine toimub konstantsel temperatuuril \( T = \SI{272}{\K} \), mille juures küllastunud veeauru rõhk \( p = \SI{550}{\Pa} \). Klaasi mõõte ja vee massi hinnake ise.
\end{question}
\begin{question}[NBPhO 2018, P8]
Tundmatut päritolu allveelaev sõidab Läänemere põhja lähedal sügavusel \( h = \SI{300}{\m} \). Selle sisemuseks on üks suur kamber ruumalaga \( V = \SI{10}{\m\tothe{3}} \), mida täidab õhk (\( M = \SI{29}{\g\per\mol} \)) rõhul \( p_0 = \SI{100}{\kPa} \) ja temperatuuril \( t_0 = \SI{20}{\degreeCelsius} \). Ootamatult sõidab ta aga karile, mis rebib laeva põhja suure augu pindalaga \( A = \SI{20}{\cm\squared}\). Allveelaev vajub põhja ja täitub kiiresti veega, välja arvatud üks kõrgemal rõhul õhumull (kogu õhk jääb laeva). Vee tihedus \( \rho_0 = \SI{1000}{\kg\per\m\cubed} \) ja raskuskiirendus \( g =\SI{9,81}{\m\per\s\squared} \). Õhu molaarne erisoojus jääval ruumalal \( c_V =\frac{5}{2}R \), kus \( R = \SI{8,31}{\J\per\mol\per\K}  \) on gaasikonstant.
\begin{subquestion}
	\item Millisel mahukiirusel (\si{\m\cubed\per\s}) voolab vesi laeva vahetult pärast augu tekkimist?
	\item Veevool on niivõrd kiire, et gaasi ja vee vaheline soojusvahetus on tühine (see kehtib ka järgmise küsimuse jaoks). Kui suur on õhumulli ruumala, kui veevool peatub?
	\item Laeva paiskuv veejuga paneb laevasoleva vee turbulentselt liikuma. Kui suur on selle veeturbulentsi summaarne kineetiline energia (mis hiljem eraldub soojusena), kui sissevoolrõhkude tasakaalustumisel peatub?
\end{subquestion}
\end{question}
\begin{question}[EuPhO 2017, T2]
	Õhuke ketas massiga \( M \) ja pindalaga \( S \) temperatuuril \( T_1 \) on alguses paigal kaalutus olekus gaasis massitihedusega \( \rho \) temperatuuriga \( T_0 \) \(T_1=1000T_0\). Üks ketta külg on kaetud termaalselt isoleeritud kihiga, teisel küljel on on väga hea termaalne kontakt seda ümbritseva gaasiga: gaasi molekulid massiga \( m \) omandavad ketta temperatuuri ühe põrkega ketta pinnalt. Hinnake ketta algset kiirendust \( a_0 \) ja maksimaalset kiirust \( v_{max} \) järgneva liikumise jooksul. Eeldage, et ketta soojusmahtuvus suurusjärk on \( Nk_B \), kus \( N \) on selles olevate aatomite arv ja \( k_B \) on Boltzmanni konstant, ja gaasi molekulide ja ketta materjali molaarmassid on samas suurusjärgus. Molekulide keskmine vaba tee pikkus on palju suurem ketta mõõtmetest. Ärge arvestage ääreefektidega, mis tulenevad ketta äärest.
\end{question} 

\end{document}
