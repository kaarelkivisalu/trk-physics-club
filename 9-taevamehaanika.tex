\documentclass[a4paper,11pt,twocolumn]{article}
\input{structure.tex}

\begin{document}
{\huge \textbf{Taevamehaanika} \hfill \normalsize{nr 9} \\
{Kaarel Kivisalu \hfill 2. aprill 2019}

\section{Taevamehaanika}
Kepleri seadused:
\vspace{-1em}
\begin{enumerate}
    \item Iga planeedi $a_1$ orbiit on ellips, mille ühes fookuses on Päike:
        \[
            r=\frac{p}{1+e \cos f}
        .\]
    \item Planeedi raadiusvektor katab võrdsete ajavahemike jooksul võrdsed pindalad:
        \begin{align*}
            \frac{\text{d} \vec{L}}{\text{d}t}&= \frac{d(\vec{r}\times m \vec{v})}{dt}= 0  \\
            \frac{\text{d} \vec{A}}{\text{d}t}&= \frac{1}{2}\frac{\vec{L}}{m}=\text{const}
        .\end{align*}
    \item Planeetide tiirlemisperioodide ruudud suhtuvad nagu nende orbiitide pikemate pooltelgede kuubid:
        \[
            P^2=\frac{4\pi^2}{G(m_1+m_2)}a^3
        .\]
\end{enumerate}

Newtoni gravitatsiooniseadus:
\[
F=\frac{Gm_1m_2}{r^2}
.\]
Energia orbiidil:
\[
E=K+P=\frac{mv^2}{2}-\frac{Gm_1m_2}{r}=-\frac{Gm_1m_2}{2a}
.\]
Runge-Lenz'i (ekstsentrilisuse) vektor:
\[
\vec{e}=\frac{\vec{v}\times \vec{L}}{Gm_1m_2}-\hat{r}=\text{const}
.\]
Orbiite on olemas kolme tüüpi: elliptilised, hüperboolsed ja paraboolsed.
\subsection{Ellips}
Ristkoordinaatides:
\[
\frac{x^2}{a^2}+\frac{y^2}{b^2}=1
.\]
Polaarkoordinaatides:
\[
r=\frac{p}{1+r \cos f}
.\]
$a$ on suur pooltelg, väike pooltelg $b=a\sqrt{1-e^2}$, $e$ on ektsentrilisus ($0\le e<1$), fookuse kaugus keskpunktist $c=ea$, poolfokaalparameeter $p=a(1-e^2)$, pindala $A=\pi ab$, energia $E<0$.
\begin{figure}[h!]
    \centering
    \includegraphics[height=4cm]{ast1.png}
    \label{fig:ast1-png}
\end{figure}

\subsection{Hüperbool}
Ristkoordinaatides
\[
\frac{x^2}{a^2}-\frac{y^2}{b^2}=1
.\]
Polaarkoordinaatides:
\[
r=\frac{p}{1+e \cos f}
.\]
Ekstsentrilisus $r>1$, väike pooltelg $b=a\sqrt{e^2-1}$, poolfokaalparameeter $p=a(e^2-1)$, asümptoodid $y=\pm \frac{b}{a}x$, energia $E>0$.
\begin{figure}[h!]
    \centering
    \includegraphics[height=4.2cm]{ast2.png}
    \label{fig:ast2-png}
\end{figure}

\subsection{Parabool}
Ristkoordinaatides:
\[
x=-ay^2
.\]
Polaarkoordinaatides:
\[
r=\frac{p}{1+e \cos f}
.\]
Ekstsentrilisus $e=1$, poolfokaalparameeter  $p=\frac{1}{2}a$, $h=\frac{1}{4}a$, energia $E=0$.
\begin{figure}[h!]
    \centering
    \includegraphics[height=5cm]{ast3.png}
    \label{fig:ast3-png}
\end{figure}
\vspace{-1em}

\subsection{Ülesanded}

\begin{question}
    Leidke ringorbiidil raadiusega $R=R_M$ liikumise periood, kus $R_M$ on Maa raadius.
\end{question}

\begin{question}
    Leidke aeg, mis kulub Maa sisse kaevatud tsentrit läbiva tunneli kaudu \enquote{kukkumiseks} teisele poole Maakera.
    \begin{hint}
    Vihje: kiirenduse sõltuvus koordinaadist peaks tulema harmoonilise ostsillaatoriga samal kujul $a=-\omega^2x$. Võrrelge seda aega esimese kosmilise kiirusega Maa ümber tiiru tegemise ajaga. Kas tulemus oleks teine, kui see tunnel ei läbiks Maa keskpunkti?
    \end{hint}
\end{question}

\begin{question}
    Tõestage, et kui planeedi sisemusse teha sfääri-kujuline õõnsus, siis selle sees on väli homogeenne.
\end{question}

\begin{question}
    Tuletage koguenergia (kineetiline pluss potentsiaalne) avaldis
    \[
    E=-\frac{Gm_1m_2}{2a}
    .\]
\end{question}

\begin{question}
    Maapinnalt visatakse vertikaalselt üles mingi objekt sellise kiirusega, et see eemaldub Maast kaugusele $R$ ning tuleb siis tagasi. Leidke objekti lennu-aeg. Maa raadius on $R_M=R$.
\end{question}

\begin{question}
    Objekt saadetakse orbiidile kahes etapis: esmalt antakse maapinna läheduses kiirus $v_2$ ning apogees suurendatakse kiirust: $v_2=v+\Delta v$ nii, et objekt on nüüd ringorbiidil raadiusega $R$, Maa raadius on $R_M$. Leida $v_1$ ja $\Delta v$.
\end{question}

\begin{question}
   Ringorbiidil $R$ objektile antakse radiaalsuunaline kiirus. Kui suur peab see kiirus olema, et objekt väljuks Maa orbiidilt? Maa raadius on $R_M$.
   \begin{hint}
       Milline peab olema objekti koguenergia, et väljuda orbiidilt (liikuda lõpmatult kaugele)?
   \end{hint}
\end{question}

\begin{question}
    Mingi tähe läheduses toimub plahvatus, kust lendab välja palju väikseid objekte, kõigi kiiruste moodulid on võrdsed $v$-ga. Objektid hakkavad liikuma elliptilisi orbiite mööda, mille üheks fookuseks on see täht. Kus võib aga asuda teine fookus? (Leidke teise fookuse geomeetriline koht.)
\end{question}

\begin{question}[Tudengite olümpiaad 2003]
    Vaadelgem tähtedevahelise gaasipilve gravitatsioonilist kokkutõmbumist. Eeldagem, et gaas tihedusega $\rho=\SI{10e-15}{kg.m^{-3}}$ täidab ühtlaselt kerakujulise ruumiosa ning algtemperatuur on nii madal, et osakeste algkiiruse võib lugeda nulliks. Kui kaua võtab aega gaasipilve kokkutõmbumine?
\end{question}

\begin{question}[ast4][0.9\columnwidth]
    Lõpmatusest läheneb tähele mingi objekt ning eemaldub seejärel taas lõpmatusse. Lõpmatult kaugel olles on kiirusvektori poolt moodustatd sirge kaugus tähest $b$ (vt joonist), pärast eemaldumist on sama parameeter $b'$ (milline on seos nende vahel?). Leidke avaldis kõrvalekalde-nurga $\phi$ jaoks.
    \begin{hint}
        Runge-Lenz'i vektor võib osutuda kasulikuks.
    \end{hint}
\end{question}

\end{document}
