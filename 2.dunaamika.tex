\documentclass[a4paper,11pt,twocolumn]{article}
\usepackage{polyglossia} % Eesti keele tugi
\setdefaultlanguage{estonian}
\usepackage{geometry}% Paigutus
\usepackage{graphicx}% Joonised
\graphicspath{ {images/} }
\usepackage{csquotes}% Eesti jutumärgid \enquote{}
\usepackage{enumitem}% Listid
\usepackage[compact]{titlesec}% Kompaktsed pealkirjad
\usepackage{siunitx}% SI ühikud
\usepackage{tikz}
\usepackage[siunitx]{circuitikz}
\usepackage[final]{microtype}
\usepackage{amsmath}
\usepackage{lmodern}
\geometry{
    paper=a4paper, % Paper size, change to letterpaper for US letter size
    top=0.5cm, % Top margin
    bottom=1cm, % Bottom margin
    left=0.5cm, % Left margin
    right=0.5cm, % Right margin
    foot=0.5cm, % Footer-margin distance
    %showframe, % Uncomment to show how the type block is set on the page
}


%\setlength{\intextsep}{0pt}

\setlength{\parindent}{0cm}% Taandridu pole
\setlength{\parskip}{1em}% Paragraafide vahed
\setlist[itemize]{topsep=0em, partopsep=0em, parsep=0em, itemsep=0.5em}% Itemize spacing

\usepackage{xparse}

% Ülesanded \begin{question}[viide][joonis][joonise suurus] (võib olla ka ainult [viide] või [joonis][joonise suurus])
\newcounter{myproblems}
\NewDocumentEnvironment{question}{o o o}
{\par \refstepcounter{myproblems} \textbf{Ülesanne \themyproblems .} \ignorespaces\IfValueT{#1}{\IfValueTF{#3}{\textbf{(#1)} \ignorespaces}{\IfNoValueT{#2}{\textbf{(#1)} \ignorespaces}}}}
{\IfValueT{#2}{\IfValueTF{#3}{\begin{figure}[h!]\includegraphics[width=#3]{#2.pdf}\centering\vspace{-1em}\end{figure}}{\begin{figure}[h!]\includegraphics[width=#2]{#1.pdf}\centering\vspace{-1em}\end{figure}}}\ignorespacesafterend}

% Alaülesanded
\newenvironment{subquestion}
{\setlength{\parskip}{0pt}\begin{enumerate}[label=\alph*), nolistsep]}
{\end{enumerate}\setlength{\parskip}{1em}\ignorespacesafterend}

% Vihjete jaoks
\newenvironment{hint}[1][Vihje]
{\setlength{\parskip}{0em} \textit{#1}: \ignorespaces}
{\setlength{\parskip}{1em}\ignorespacesafterend}

\newcommand{\pvec}[1]{\vec{#1}\mkern2mu\vphantom{#1}}% Primed vector

% Lahendused jaoks
\usepackage{hyperref}
\newenvironment{solutions}
{\begin{enumerate}[label=\textbf{\arabic*.}, wide]}
{\end{enumerate}}

% Displaystyle valemite paigutus
\makeatletter
\g@addto@macro{\normalsize}{%
    \setlength{\abovedisplayskip}{4pt}
    \setlength{\abovedisplayshortskip}{4pt}
    \setlength{\belowdisplayskip}{4pt}
    \setlength{\belowdisplayshortskip}{4pt}
    }
\makeatother

% \directlua{dofile("DetectUnderfull.lua")}
\tikzset{
    odot/.style={
        circle,
        inner sep=0pt,
        node contents={$\odot$},
        scale=1
    },
    otimes/.style={
        circle,
        inner sep=0pt,
        node contents={$\otimes$},
        scale=1
    }}


\begin{document}
{\huge \textbf{Dünaamika} \hfill \normalsize {nr 2.0.1}} \\
{Kaarel Kivisalu \hfill 19. september 2018}\\

\section*{Sissejuhatus}
Suur osa dünaamika ülesannetest seisneb selles, et on vaja leida mingitest kehadest koostatud süsteemi kiirendus või nende kehade vahel mõjuvad jõud. Neid ülesandeid on võimalik lahendada harilikult mitut moodi, kuid välja võiks tuua kolm meetodit:
\begin{itemize}
	\item Leiame ning paneme kirja iga keha jaoks kõik talle mõjuvad jõud, sh rõhumisjõud ja hõõrdejõud ning kirjutame välja Newtoni II seaduse komponentide kujul.
	\item Muus osas sama, mis eelmine meetod, kuid
	liikumist vaadeldakse mitteinertsiaalses taustsüsteemis
	(vt staatika materjali), kus üks kehadest on paigal.
	\item Kui süsteemi olek on kirjeldatav üheainsa arvu abil, siis nimetagem seda üldistatud koordinaadiks $ \xi $. Olgu meil vaja leida koordinaadi $ \xi $ kiirendus $ \ddot{\xi} $. Kui meil õnnestub avaldada süsteemi potentsiaalne energia $ \Pi $ koordinaadi $ \xi $ funktsioonina $ \Pi(\xi) $ ja kineetiline energia kujul $ K = \mathcal{M}\dot{\xi}^2/2 $, kus kordaja $ \mathcal{M} $ on kombinatsioon keha massidest (ja võib-olla ka inertsimomentidest), siis$$ \ddot{\xi}=−\Pi'(\xi)/\mathcal{M}.$$Siinjuures ülapunkt tähistab tuletist aja järgi ning primm tuletist koordinaadi $ \xi $ järgi. Tõepoolest, energia jäävusest tulenevalt $\Pi(\xi) + \mathcal{M}\dot{\xi}^2/2=$ Const. Võttes siit tuletise aja järgi 	ja kasutades liitfunktsiooni diferentseerimise reeglit saame $\Pi'(\xi)\dot{\xi} + \mathcal{M}\dot{\xi}\ddot{\xi}=0$. Taandades $ \dot{\xi} $ jõuamegi eelpooltoodud valemini.
\end{itemize}

\section{Soojenduseks}
\begin{question}[Piirk 2018, P3][dun7][3.5cm]
	Kaks kuulikest alustavad samaaegselt võrdsete kiirustega liikumist mööda joonisel näidatud pindu. Kumma kuulikese kiirus on punkti $ B $ jõudes suurem või on nende kiirused võrdsed? Kumb kuulike jõuab punkti $ B $ varem või jõuavad nad punkti $ B $ samaaegselt? Hõõrdejõudu pole vaja arvestada. Põhjendage vastust.
\end{question}
\begin{question}[Lahtine 2014, N4]
	Kaubarongi massiga $ m = 5000 $ t veab vedur võimsusega $ P = 2500 $ kW. Veerehõõrdetegur rataste ja rööpa vahel on $ \mu = 0,002 $. Õhutakistusega mitte arvestada.
	\begin{subquestion}
		\item Leidke rongi kiirus $ v_1 $ horisontaalsel teel.
		\item Leidke rongi kiirus $ v_2 $ tõusul üks sentimeeter ühe meetri kohta.
	\end{subquestion}	
\end{question}

\section{Newtoni II seadus $ \vec{F}=d\vec{p}/dt $ ja pöördliikumise teoreemid}
Kui keha mass jääb muutumatuks, omandab see seadus tuttava kuju $ \vec{F}=m\vec{a} $. Muutuva massi korral tuleb aga arvestada, et $ \vec{F}dt=d\vec{p}=\vec{v}dm+md\vec{v} $. Sarnaselt seadusele $ \vec{F}=d\vec{p}/dt $ kehtib ka seadus $ \vec{\tau}=d\vec{L}/dt $, mis ütleb, et süsteemile mõjuv jõumoment on võrdne impulsimomendi muutumise kiirusega.
\begin{question}
	Lähtudes inertsimomendi definitsioonist $ I=\sum m_i r_i^2 $, avaldage varda inertsimoment ümber otspunkti $ I=\frac{1}{3}ml^2 $ ning ketta inertsimoment ümber keskpunkti $ I=\frac{1}{2}ml^2 $. (Kera inertsimoment ümber keskpunkti on $ I=\frac{2}{5}ml^2 $, aga seda me praegu ei tõesta.)
\end{question}
\begin{question}
	Kui keha inertsimoment ümber massikeset läbiva telje on $ I_0 $, siis ümber mingi telje, mis paikneb sellest kaugusel b, on inertsimoment $ I=I_0+mb^2 $. Tõestage see tulemus, mis on ühtlasi tundtud ka Steineri teoreemi nime all (ingl k \textit{parallel axis theorem}).
\end{question}
\begin{question}
Lähtudes impulsimomendi definitsioonist $ L=\sum r_im_iv_{\bot i} $, näidake, et massikeskme suhtes kehtib jäiga keha jaoks $ L=I\omega $, kus $ \omega $ on pöörlemise nurkkiirus. Muuhulgas tuleneb siit muutumatu inertsimomendi korral seadus $ \tau =Id\omega/dt$
\end{question}
\begin{question}
	Lähtudes kineetilise energia definitsioonist $ E=\sum \frac{1}{2}m_i v_i^{2}$, näidake, et pöörlemise kineetiline energia jäiga keha jaoks avaldub kujul $ E=\frac{1}{2}I\omega^{2} $.	
\end{question}
\begin{question}
	Ketti pikkusega $ L $ ning joontihedusega $ \sigma $ hoitaks vertikaalselt rippuvana kaalu kohal nii, et selle alumine ots vaevalt puutub vastu kaalu. Seejärel lastakse kett vabalt kukkuma. Milline on kaalu näit sõltuvana vertikaalse keti pikkusest y? Milline on kaalu näit hetkel, mil viimane keti osa langeb kaalule?
\end{question}

\section{Ülesanded}
\begin{question}[Piirk 2015, G6][dun2][3.5cm]
Algselt paigal olev rippuv varras massiga $ M $ ning pikkusega $ L $ on fikseeritud ülemisest otsast vabalt pöörleva kinnitusega. Varda inertsimoment otspunkti suhtes on $ I = \frac{1}{3}ML^2 $ . Teraspall massiga $ m $ lendab vastu varrast ning tabab seda kaugusel $ h $ riputuspunktist. Põrge on elastne, st soojuskadudeta. Huvitaval kombel jääb teraskuul pärast põrget hetkeks paigale ning hakkab seejärel vertikaalselt alla langema. Leidke kauguse $ h $ väärtus, mille korral niisugune seismajäämine võimalik on.
\end{question}
\begin{question}[dun1][3.8cm]
	Klots massiga $ M $ lebab libedal horisontaalpinnal. Tema peale on asetatud klots massiga $ m $, mille külge on kinnitatud nööri abil teine samasugune klots. Nöör on tõmmatud üle suure klotsi nurgas asuva ploki ning teine väike klots ripub alla. Algselt hoitakse süsteemi paigal. Leidke suure klotsi kiirendus vahetult peale süsteemi vabastamist. Hõõre lugeda kõikjal nulliks, nööri ja ploki massi mitte arvestada.
\end{question}
\begin{question}[dun4][5.5cm]
Hästi kergest ja libedast materjalist on valmistatud klots, mille ülapind koosneb kahest üksteise poole pööratud kaldpinnast kaldenurgaga $ \alpha $. Klots asub horisontaalpinnal; klotsi ülapinnal moodustuva vao põhjas lebab kuul massiga $ m $ ning sellest kõrgemale asetatakse kuul massiga $ M $. Süsteem lastakse vabalt liikuma. Millisel tingimusel hakkab kuulike massiga $ m $ mööda kaldpinda ülespoole libisema? Hõõrdejõududega mitte arvestada.
\end{question}
\begin{question}[dun3][5.5cm]
	Kiilul nurgaga $ \alpha $ ja massiga $ M $ asub klotsike massiga $ m $, mis on kinnitatud üle kiilu tipus oleva ploki tõmmatud nööriga horisontaalseina külge nii, nagu näidatud joonisel. Leidke, millise kiirendusega hakkab liikuma kiil. Kõik  pinnad on libedad (hõõre puudub).
\end{question}
\begin{question}[dun5][6cm]
	Horisontaalpinnal lebab kiil massiga M ja teravnurkadega $ \alpha_1 $ ning $ \alpha_2 $. Üle kiilu tipus asuva ploki on visatud nöör, mille otstes on klotsid massidega $ m_1 $ ja $ m_2 $. Millise kiirendusega hakkab liikuma kiil? Hõõre on igal pool null.
\end{question}
\begin{question}[dun8][6cm]
	Raske telje otstes on kerged rattad raadiusega $ R $. Süsteem veereb mööda horisontaalpinda, mis läheb järsku üle kaldpinnaks kaldenurgaga $ \alpha $ moodustades nürinurkse jõnksu. Millise nurga $ \alpha $ puhul liiguvad rattad ilma õhku hüppamata, so puutuvad kogu aeg vastu pinda? Rataste massi mitte arvestada, telg on paralleelne horisontaal- ja kaldpinna eraldusservaga. Telje kiirus on $ v $.
\end{question}

\section{Keerulisemad ülesandeid}
\begin{question}[dun6][7cm]
	Kolm pisikest silindrikest on ühendatud kaalutute varraste abil, kusjuures keskmise silindri juures on liigend, nii et varraste vaheline nurk saab vabalt muutuda; alghetkel on see täisnurk. Kaks silindrit on ühesuguse massiga $ m $, ühe otsmise silindri mass on $ 4m $ aga peale liikuma hakkamist. Hõõre on igal pool null.
\end{question}
\begin{question}[E-S 2015, P7]
	Homogeenne elastne pall raadiusega $ R $ põrkas vertikaalsest seinast tagasi samas suunas, kust ta tuli. Palli kiirus enne põrget on $ v $ ja see on vertikaali suhtes nurga $ \alpha $ all. Pallil on enne põrget nurkkiirus $ \omega $. Palli kontaktpunkt seinaga ei libisenud põrke ajal, aga ärge eeldage, et $ \omega R = v \cos \alpha $. Põrge oli täiesti elastne, st kogu kineetiline energia jäi alles, samuti jäi samaks kiiruse horisontaalse komponendi suurus. Homogeense palli intertsimoment on $ I = \frac{2}{5} mR^2 $. 
	\begin{subquestion}
		\item Leidke palli nurkkiirus $ \omega_2 $ pärast põrget.
		\item Leidke palli esialgne nurkkiirus $ \omega $ (avaldage $ v $, $ \alpha $ ja $ R $ abil).
		\item Milline on minimaalne väärtus hõõrdetegurile $ \mu $, et selline põrge saaks toimuda?
	\end{subquestion} 
\end{question}
\begin{question}[EuPhO 2018, T1]
	Kolm väikest ühesugust palli (märgitud kui $ A $, $ B $ ja $ C $) massiga $ m $ on ühendatud kahe massitu vardaga pikkusega $ l $. Pallid $ A $ ja $ B $ ning $ B $ ja $ C $ on omavahel ühendatud. Ühendus palli B juures on šarniirne. Süsteem on paigal kaalutus olekus nii, et kõik pallid on ühel sirgel. Pallile $ A $ antakse hetkeliselt varrastega risti kiirus $ v $. Leia minimaale kaugus $ d $ pallide $ A $ ja $ C $ vahel järgneva liikumise jooksul. Hõõrdumisega mitte arvestada.
\end{question}

\end{document}