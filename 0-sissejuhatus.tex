\documentclass[a4paper,11pt,twocolumn]{article}
\usepackage{polyglossia} % Eesti keele tugi
\setdefaultlanguage{estonian}
\usepackage{geometry}% Paigutus
\usepackage{graphicx}% Joonised
\graphicspath{ {images/} }
\usepackage{csquotes}% Eesti jutumärgid \enquote{}
\usepackage{enumitem}% Listid
\usepackage[compact]{titlesec}% Kompaktsed pealkirjad
\usepackage{siunitx}% SI ühikud
\usepackage{tikz}
\usepackage[siunitx]{circuitikz}
\usepackage[final]{microtype}
\usepackage{amsmath}
\usepackage{lmodern}
\geometry{
    paper=a4paper, % Paper size, change to letterpaper for US letter size
    top=0.5cm, % Top margin
    bottom=1cm, % Bottom margin
    left=0.5cm, % Left margin
    right=0.5cm, % Right margin
    foot=0.5cm, % Footer-margin distance
    %showframe, % Uncomment to show how the type block is set on the page
}


%\setlength{\intextsep}{0pt}

\setlength{\parindent}{0cm}% Taandridu pole
\setlength{\parskip}{1em}% Paragraafide vahed
\setlist[itemize]{topsep=0em, partopsep=0em, parsep=0em, itemsep=0.5em}% Itemize spacing

\usepackage{xparse}

% Ülesanded \begin{question}[viide][joonis][joonise suurus] (võib olla ka ainult [viide] või [joonis][joonise suurus])
\newcounter{myproblems}
\NewDocumentEnvironment{question}{o o o}
{\par \refstepcounter{myproblems} \textbf{Ülesanne \themyproblems .} \ignorespaces\IfValueT{#1}{\IfValueTF{#3}{\textbf{(#1)} \ignorespaces}{\IfNoValueT{#2}{\textbf{(#1)} \ignorespaces}}}}
{\IfValueT{#2}{\IfValueTF{#3}{\begin{figure}[h!]\includegraphics[width=#3]{#2.pdf}\centering\vspace{-1em}\end{figure}}{\begin{figure}[h!]\includegraphics[width=#2]{#1.pdf}\centering\vspace{-1em}\end{figure}}}\ignorespacesafterend}

% Alaülesanded
\newenvironment{subquestion}
{\setlength{\parskip}{0pt}\begin{enumerate}[label=\alph*), nolistsep]}
{\end{enumerate}\setlength{\parskip}{1em}\ignorespacesafterend}

% Vihjete jaoks
\newenvironment{hint}[1][Vihje]
{\setlength{\parskip}{0em} \textit{#1}: \ignorespaces}
{\setlength{\parskip}{1em}\ignorespacesafterend}

\newcommand{\pvec}[1]{\vec{#1}\mkern2mu\vphantom{#1}}% Primed vector

% Lahendused jaoks
\usepackage{hyperref}
\newenvironment{solutions}
{\begin{enumerate}[label=\textbf{\arabic*.}, wide]}
{\end{enumerate}}

% Displaystyle valemite paigutus
\makeatletter
\g@addto@macro{\normalsize}{%
    \setlength{\abovedisplayskip}{4pt}
    \setlength{\abovedisplayshortskip}{4pt}
    \setlength{\belowdisplayskip}{4pt}
    \setlength{\belowdisplayshortskip}{4pt}
    }
\makeatother

% \directlua{dofile("DetectUnderfull.lua")}
\tikzset{
    odot/.style={
        circle,
        inner sep=0pt,
        node contents={$\odot$},
        scale=1
    },
    otimes/.style={
        circle,
        inner sep=0pt,
        node contents={$\otimes$},
        scale=1
    }}


\begin{document}
{\huge \textbf{Sissejuhatus}\hfill \normalsize {nr 0}} \\
{Kaarel Kivisalu \hfill \today}

\section{Materjalid}

\begin{itemize}
\item Füüsikaklubi materjalid: \url{https://github.com/kaarelkivisalu/trk-physics-club}
\item EFO koduleht: \url{http://efo.fyysika.ee/}
\item Ettevalmistus rahvusvahelisteks olümpiaadideks: \url{https://www.ioc.ee/~kalda/ipho/}
\item Füüsikaolümpiaad: \url{https://www.teaduskool.ut.ee/et/olumpiaadid/fuusikaolumpiaad}
\item Füüsika lahtine: \url{https://www.teaduskool.ut.ee/et/olumpiaadid/fuusika-lahtine}
\item Kajakas: \url{https://www.teaduskool.ut.ee/et/ainevoistlused/kajakas}
\item Teaduskooli õppematerjalid: \url{https://www.teaduskool.ut.ee/et/oppetoo/fuusika-oppematerjalid}
\item Õpikojad: \url{https://www.teaduskool.ut.ee/et/oppetoo/opikojad}
\end{itemize}

\section{Ajakava}

\begin{tabular}{l l l}
    Füüsika lahtine & 21. november 2020 \\
    EFO koolivoor & ? \\
    EFO piirkonnavoor & 6. veebruar 2021 \\
    EFO lõppvoor & 10.--11. aprill 2021 \\
    EuPhO & ? & (Rumeenia)\\
    NBPhO & ?\\
    IPhO & 17.-25. juuli 2021 & Leedu\\
    Astronoomia lahtine & 11. aprill 2021 \\
    IOAA & ? & Columbia\\
    Kajakas & ? \\
\end{tabular}

\section{Füüsikaklubist}

\subsection{Füüsikaklubi presidendid}
\begin{tabular}{l l l}
    Kaarel Hänni & 2016/17 & 132. a \\
    Kaarel Kivisalu & 2017/18 -- 2019/20 & 135. a \\
    ? & 2020/21 -- ? & ? \\
\end{tabular}


\end{document}
