\documentclass[a4paper,11pt,twocolumn]{article}
\input{structure.tex}

\begin{document}
{\huge \textbf{Sissejuhatus}\hfill \normalsize {nr 0}} \\
{Kaarel Kivisalu \hfill \today}

\section{Füüsikaklubist}

Toimub kolmapäeviti alates 14:45 kuni inimesed laiali lähevad Reaalkooli 2. korruse vabas klassiruumis.

\subsection{Füüsikaklubi presidendid}
\begin{tabular}{l l l}
    Kaarel Hänni & 2016/17 & 132. a \\
    Kaarel Kivisalu & 2017/18 -- 2019/20 & 135. a \\
    ? & 2020/21 -- ? & ? \\
\end{tabular}

\section{Materjalid}

\begin{itemize}
\item Füüsikaklubi materjalid: \url{https://github.com/kaarelkivisalu/trk-physics-club}
\item EFO koduleht: \url{http://efo.fyysika.ee/}
\item Ettevalmistus rahvusvahelisteks olümpiaadideks: \url{https://www.ioc.ee/~kalda/ipho/}
\item Füüsikaolümpiaad: \url{https://www.teaduskool.ut.ee/et/olumpiaadid/fuusikaolumpiaad}
\item Füüsika lahtine: \url{https://www.teaduskool.ut.ee/et/olumpiaadid/fuusika-lahtine}
\item Kajakas: \url{https://www.teaduskool.ut.ee/et/ainevoistlused/kajakas}
\item Teaduskooli õppematerjalid: \url{https://www.teaduskool.ut.ee/et/oppetoo/fuusika-oppematerjalid}
\item Õpikojad: \url{https://www.teaduskool.ut.ee/et/oppetoo/opikojad}
\end{itemize}

\section{Ajakava}

\begin{tabular}{l l l}
    Füüsika lahtine & 21. november 2020 \\
    EFO koolivoor & ? \\
    EFO piirkonnavoor & 6. veebruar 2021 \\
    EFO lõppvoor & 10.--11. aprill 2021 \\
    EuPhO & ? & (Rumeenia)\\
    NBPhO & ?\\
    IPhO & 17.-25. juuli 2021 & Leedu\\
    Astronoomia lahtine & 11. aprill 2021 \\
    IOAA & ? & Columbia\\
    Kajakas & ? \\
\end{tabular}

\end{document}
